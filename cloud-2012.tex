\documentclass[10pt, conference, compsocconf]{IEEEtran}

\ifCLASSINFOpdf
  \usepackage[pdftex]{graphicx}
  \DeclareGraphicsExtensions{.pdf,.jpeg,.jpg,.png,.pdf}
\else
  \usepackage[dvips]{graphicx}
  \DeclareGraphicsExtensions{.eps}
\fi

\graphicspath{{./images/}}

\usepackage{amsmath}
\usepackage{amsfonts}
\usepackage{amssymb}
\usepackage{setspace}
\usepackage{graphicx}
\usepackage{epsfig}
\usepackage{booktabs}

\hyphenation{op-tical net-works semi-conduc-tor}

\begin{document}

\title{Usage Management Overlay Architectures: Paths from Cross Domain Solutions}

\author{\IEEEauthorblockN{Christopher C. Lamb}
\IEEEauthorblockA{Department of Electrical and Computer Engineering\\
University of New Mexico\\
Albuquerque, NM, USA\\
cclamb@ece.unm.edu}
\and
\IEEEauthorblockN{Gregory L. Heileman}
\IEEEauthorblockA{Department of Electrical and Computer Engineering\\
University of New Mexico\\
Albuquerque, NM, USA\\
heileman@ece.unm.edu}
}

\maketitle


\begin{abstract}
Herein, we contrast filter-centric diode and policy-centric overlay approaches toward usage management of sensitive content from the perspective of cross-domain solutions and then demonstrate a path from current systems to distributed policy-centric overlay methods.  We first cover the need for overlay networks supporting usage management by highlighting the pending migration to utility computing models and the current shortcomings of those kinds of systems from a variety of perspectives.  We also briefly cover the state of usage management within these kinds of systems, also examining the known state of the art in proposed cross domain solutions.  The then compare and contrast the advantages of policy-based overlay systems to solving usage management issues over sensitive material and demonstrate a migration path from current to future solutions.  We close with our current and future directions of research. 
\end{abstract}

\begin{IEEEkeywords}
usage management; overlay networks; quantitative architecture evaluation;

\end{IEEEkeywords}

\IEEEpeerreviewmaketitle

\section{Introduction}
Current enterprise computing systems are facing a troubling future.  As things stand today, they are too expensive, unreliable, and information dissemination procedures are just too slow.

Generally, such systems still do not use current commercial resources as well as they could and use costly data partitioning schemes.  Most of these kinds of systems use some combination of systems managed in house by the enterprise itself rather than exploiting lower cost cloud-enabled services.  Furthermore, many of these systems have large maintenance loads imposed on them as a result of internal infrastructural requirements like data and database management or systems administration.  In many cases networks containing sensitive data are separated from other internal networks to enhance data security at the expense of productivity, leading to decreased working efficiencies and increased costs.

These kinds of large distributed systems suffer from a lack of stability and reliability as a direct result of their inflated provisioning and support costs.  Simply put, the large cost and effort burden of these systems precludes the ability to implement the appropriate redundancy and fault tolerance in any but the absolutely most critical systems.  Justifying the costs associated with standard reliability practices like diverse entry or geographically separated hot spares is more and more difficult to do unless forced by draconian legal policy or similarly dire business conditions.

Finally, the length of time between when a sensitive document or other type of data artifact is requested and when it can be delivered to a requester with acceptable need to view that artifact is prohibitively long.  These kinds of sensitive artifacts, usually maintained on partitioned networks or systems, require large amounts of review by specially trained reviewers prior to release to data requesters.  In cases where acquisition of this data is under hard time constraints like sudden market shifts or other unexpected conditional changes this long review time can result in consequences ranging from financial losses to loss of life.

Federal computer systems are prime examples of these kinds of problematic distributed systems, and demonstrate the difficulty inherent in implementing new technical solutions.  They, like other similar systems, need to be re-imagined to take advantage of radical market shifts in computational provisioning.

\section{Motivation}
Current policy-centric systems are being forced to move to cloud environments and incorporate much more open systems.  Some of these environments will be private or hybrid cloud systems, where private clouds are infrastructure that is completely run and operated by a single organization for use and provisioning, while hybrid clouds are combinations of private and public cloud systems.  Driven by both cost savings and efficiency requirements, this migration will result in a loss of control of computing resources by involved organizations as they attempt to exploit economies of scale and utility computing.

Robust usage management will become an even more important issue in these environments.  Federal organizations poised to benefit from this migration include agencies like the National Security Agency (NSA) and the Department of Defense (DoD), both of whom have large installed bases of compartmentalized and classified data.  The DoD realizes the scope of this effort, understanding that such technical change must incorporate effectively sharing needed data with other federal agencies, foreign governments, and international organizations \cite{proposal:info-sharing-strategy}.  Likewise, the NSA is focused on exploiting cloud-centric systems to facilitate information dissemination and sharing \cite{proposal:nsa-cloud}.

Cloud systems certainly exhibit economic incentives for use, providing cost savings and flexibility, but they also have distinct disadvantages as well.  Specifically, the are not intrinsically as private as some current systems, generally can be less secure than department-level solutions, and have the kinds of trust issues that the best of therapists cannot adequately address \cite{proposal:privacy-security-trust-cloud}.

As Pearson and Benameur \cite{proposal:privacy-security-trust-cloud} show, cloud technology is not currently as private as some organizations would like:
\begin{itemize}
\item \textit{User Data Control} --- In virtually any given Software-as-a-Service (SaaS) scenario, user data controls are sadly lacking.  Once data has been committed to a specific provider, that data is completely out of the original data owners control.  Furthermore, as we will see below, that data my not even be solely owned by the original owner anymore either.
\item \textit{Secondary Use} --- Most consumer facing social systems extensively mine user provided data for additional business advantages.  This is a common and well known secondary use for supplied data.  SaaS providers again have strong incentives to examine user provided information.
\item \textit{Offshore Development} --- Service users have no real control over who actually develops the systems a given service deploys.  Organizations have attempted to contractually limit development and support functions companies pursue to, say, the continental United States but have had very poor results with these kinds of unsupportable arrangements.
\item \textit{Data Routing} --- Both system providers and system users in fact have little control over routing issues.  Prohibiting data routing through sensitive countries is a difficult task for a single organization.
\item \textit{Secondary Storage} --- Most large-scale systems expect to use Content Delivery Networks (CDNs) to help manage content, and that expectation is heavily reflected in their physical system architectures. They simply cannot divorce use of CDNs from their systems for a single organization.
\item \textit{Bankruptcy and Data Ownership} --- Ownership and obligation to maintain expected data arrangements for a given company is not established under bankruptcy \cite{proposal:borders-info-I,proposal:borders-info-II,proposal:borders-info-III}.
\end{itemize}

Security issues also emerge from utility computing infrastructures:
\begin{itemize}
\item \textit{Data Access} --- System users have very little control over who, in the system provider's organization, is able to access their data and systems.
\item \textit{Data Deletion} --- Most savvy organizations have procedures in place to sanitize old storage elements like disk drives or backup tapes.  System users have very little control over if and how this is done when computing services are treated as a utility.
\item \textit{Backup Data Storage} --- Backup media is very difficult to encrypt, and most system providers still use tape systems as preferred media solutions for backup and storage needs.  These tapes, or copies of them, are generally stored offsite to support disaster recovery scenarios.  Security of these types of systems has been spotty to date \cite{proposal:saic-breach-I,proposal:saic-breach-II,proposal:saic-breach-III}.
\item \textit{Intercloud Standardization} --- Cloud computing systems do not have any standardized way to transfer computational units or data between systems.  Any protocols used for this kind of thing must be developed by customers themselves.  Due to the desire of providers to lock-in customers, this will likely not change as any standard development is strongly counter-incentiveized. 
\item \textit{Multi-tenancy and Side-Channels} --- Multi-tenant architectures in which multiple customers simultaneously use the same systems open those customers to covert side-channel attacks.
\item \textit{Logging and Auditing} --- Logging and auditing structures, especially for inter-cloud systems, are non-existent.
\end{itemize}

Finally, such systems suffer from internal and external trust issues:
\begin{itemize}
\item \textit{Trust Relationships} --- Trust is difficult to establish between individual cloud providers long-term.
\item \textit{Consumer Trust} --- Service users are still not entirely trusting of cloud system providers.
\end{itemize}

How to address these issues is an open research question.  Organizations ranging from cloud service providers to the military are exploring how to engineer solutions to these problems, and to more clearly understand the trade-offs required between selected system architectures \cite{proposal:assured-info-sharing}.  The problems themselves are wide ranging, appearing in a variety of different systems.  Military and other government systems are clearly impacted by these kinds of trust and security issues, and they also have clear information sensitivity problems.  This, coupled with the fact that these organizations have been dealing with these issues in one form or another for decades make them very well suited for prototypical implementation and study.

Current federal standards in place to deal with these issues in this environment are managed by the Unified Cross Domain Management Office (UCDMO).  UCDMO stakeholders range from the DoD to the NSA.  The current standard architectural model in place and governed by the UCDMO to deal with this kinds of issues are \textit{guard-centric cross domain architectures}.

These kinds of cross-domain solutions still have clear similarities, and in fact have not progressed far beyond the initial notions of how these kinds of systems should work.  They still, for example, all use some kind of filter chaining mechanism to evaluate whether a given data item can be moved from a classified to an unclassified network.  Both NSA models used filters explicitly, as did the BAH model.  They all use a single guard as well, a sole point of security and enforcement, providing perimeter data security, but nothing else.  In each of these current system architectures, users are only allowed to exchange one type of information per domain.  The physical instantiations of these models are locked by operational policy to a single classification level.  Users cannot, for example, have Top Secret material on a network accredited for Secret material.  Finally, these models violate end-to-end principles in large service network design, centralizing intelligence rather than pushing that intelligence down to the ends of the system ~\cite{Clark:1995:DPD:205447.205458}.

End-to-end principles are generally considered core to the development of extreme scale, distributed systems.  Essentially, one of the key design decisions with respect to the early internet was to move any significant processing to system end nodes, keeping the core of the network fast and simple.  Known as the end-to-end principles, this design has served the internet well, allowing it to scale to sizes unconceived when originally built.  Current cross domain systems are placed at key routing points between sensitive networks.  These locations are core to information transfer between systems and as a result violate the initial design principles upon which the internet was founded.  There does exist some belief that end-to-end principles need to be modified to support future networks, but nevertheless, current cross domain systems still violate the basic ideas behind large, scalable networks by placing complex application-specific logic directly and only in the core of a given sensitive network ~\cite{Blumenthal:2001:RDI:383034.383037}.

Future systems will generally demonstrate decentralized policy management capabilities, infrastructural reuse, the ability to integrate with cloud systems, and security in depth.  Policy management will need to be decentralized and integrated within the fabric of the system.  The system is both more secure and resilient as a result, better able to control information and operate under stressful conditions.  Multi-tenancy can lower costs and increase reliability and is furthermore a common attribute of cloud systems.  An appropriately secured system facilitates integration of computing resources into multi-tenant environments.  The ability to handle multi-tenant environments and to reliably secure both data at rest and data in motion leads to computational environments deployable in cloud systems.  Finally, systems must operate under \textit{all} conditions, including when they are under attack or compromise ~\cite{proposal:ron-ross} and provide protection to sensitive data in depth.


\subsection{Cross Domain Solutions}
The Unified Cross Domain Management Office (UCDMO) supports efforts do develop other specific solutions that have been presented over the past few years within the government space to handle this kind of information management.  The National Security Agency set the standard in this area initially.  In 2009, at a conference sponsored by the UCDMO, Booz $\mid$ Allen $\mid$ Hamilton and Raytheon presented alternative notional architectures contrasting with current NSA-influenced approaches \cite{proposal:nsa-arch,proposal:gig-arch,proposal:bah-arch,proposal:raytheon-arch}.

\section{Taxonomies of Usage Management Overlay}
A clear taxonomic organization of potential steps in approaching finer grained policy based usage management helps in describing the difficulties inherent in developing potential solutions as well as aiding in planning system evolution over time. Here, we have five distinct types of integrated policy-centric usage management systems, as shown in Table \ref{table:model:taxonomy}.  Of these five, only the first two levels are represented in current system model.

\begin{table}[tp] %
\centering %
\begin{tabular}{clcc}
\toprule %
$ Name$ 	& $Description$ \\\toprule %
$\phi$ 		& The initial level of this taxonomy, $\phi$ classified systems \\
 			& have a single guard without policy-based control \\\midrule
$\alpha$	& $\alpha$ classified systems have a single guard by have begun \\
			& to integrate policy-based control \\\midrule
$\beta$		& Systems that have begun to integrate policy-based control with \\
			& router elements are in the $\beta$ category \\\midrule
$\gamma$	& Systems that have integrated policy-based control with routing \\
			& and computational elements \\\midrule
$\delta$	& Continuous policy-based control with \textit{smart licensed} artifacts \\\bottomrule
\end{tabular}
\caption{Proposed Usage Management Taxonomy}
\label{table:model:taxonomy}
\end{table}

In this taxonomy, it is not required that systems pass through lower levels to reach higher ones.  This taxonomy represents a continuum of integration of usage management controls.  Systems can very well be designed to fit into higher taxonomic categories without addressing lower categories.  That said however, many of the supporting infrastructural services, like identification management or logging and tracing systems, are common between multiple levels.

The taxonomy itself starts with the current state, integrating policy evaluation systems into the network fabric gradually, moving away from filters, then by adding policy evaluation into the routing fabric, then the computational nodes, and finally by incorporating evaluation directly into content.

\subsection{$\phi$-level Overlay Systems}

\subsection{$\alpha$-level Overlay Systems}

\subsection{$\beta$-level Overlay Systems}

\subsection{$\gamma$-level Overlay Systems}

\section{Conclusion}
The conclusion goes here. this is more of the conclusion

\bibliographystyle{IEEEtranBST/IEEEtran}
\bibliography{bib/proposal,bib/drm}

\end{document}


