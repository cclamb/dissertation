\documentclass[botnum,fleqn,final]{unmeethesis}

% Set Title and Author
\newcommand{\mytitle}{Information Protection in Content-centric Networks}
\newcommand{\myauthor}{Christopher C. Lamb}


\usepackage[latin1]{inputenc}
\usepackage{amsmath}
\usepackage{amsfonts}
\usepackage{amssymb}
\usepackage{graphicx}
\usepackage{epsfig}
\usepackage{booktabs}
\usepackage{listings}
\usepackage{courier}
\usepackage{multirow}
\usepackage{color}
\usepackage{framed}
\usepackage{lipsum}

% Font settings:
\usepackage[T1]{fontenc}
%\usepackage{mathpazo}
\usepackage{mathptmx}
\usepackage[scaled]{helvet}
\usepackage{courier}
\normalfont

% User microtype package to get some nicer font details
% http://www.ctan.org/tex-archive/macros/latex/contrib/microtype
\usepackage{microtype}

% Other packages
\usepackage{graphicx} % ability to include graphics
\usepackage[pdftex]{hyperref} % hyperrefs inside of PDF
\hypersetup{colorlinks,
            citecolor=black,
            filecolor=black,
            linkcolor=black,
            urlcolor=black,
            pdftitle={\mytitle},
            pdfauthor={\myauthor}}

% Use hyperref package to allow hyperlinks inside of 
% the PDF document and PDF metadata
% http://en.wikibooks.org/wiki/LaTeX/Hyperlinks
\usepackage[pdftex]{hyperref}
\hypersetup{
    %bookmarks=true,          % show bookmarks bar?
    unicode=false,           % non-Latin characters in Acrobat’s bookmarks
    pdftoolbar=true,         % show Acrobat’s toolbar?
    pdfmenubar=true,         % show Acrobat’s menu?
    pdffitwindow=false,      % window fit to page when opened
    pdfstartview={FitH},     % fits the width of the page to the window
    pdftitle={\mytitle},     % title
    pdfauthor={\myauthor},   % author
    pdfsubject={\mytitle},   % subject of the document
    pdfcreator={\myauthor},  % creator of the document
    pdfproducer={\myauthor}, % producer of the document
    colorlinks=true,         % false: boxed links; true: colored links
    linkcolor=black,         % color of internal links
    citecolor=black,         % color of links to bibliography
    filecolor=black,         % color of file links
    urlcolor=blue            % color of external links
}


\graphicspath{{./images/}}
 
\lstset{
	language=IDL,
	basicstyle=\scriptsize\ttfamily, 
	numbers=left,    
	numberstyle=\tiny,
	stepnumber=1,
	numbersep=5pt,
	tabsize=2,
	extendedchars=true,        
	breaklines=true,            
	keywordstyle=\color{red},
	frame=b,         
	stringstyle=\color{white}\ttfamily,
	showspaces=false,
	showtabs=false,
	xleftmargin=17pt,
	framexleftmargin=17pt,
	framexrightmargin=5pt,
	framexbottommargin=4pt,
	showstringspaces=false      
}
 
\usepackage{caption}
\DeclareCaptionFont{white}{\color{white}}
\DeclareCaptionFormat{listing}{\colorbox[cmyk]{0.43, 0.35, 0.35,0.01}{\parbox{\textwidth}{\hspace{15pt}#1#2#3}}}
\captionsetup[lstlisting]{format=listing,labelfont=white,textfont=white, singlelinecheck=false, margin=0pt, font={bf,footnotesize}}

\begin{document}

\frontmatter

\title{\mytitle}
\author{\myauthor}

\degreesubject{Ph.D., Computer Engineering}

\degree{Doctor of Philosophy \\ Computer Engineering}

\documenttype{Dissertation}

\previousdegrees{B.S., Mechanical Engineering, New Mexico State University, 1994 \\
                 M.S., Computer Science, University of New Mexico, 2002}

\date{\today}

\maketitle

\makecopyright

\begin{dedication}
This dissertation is dedicated first and foremost to my wife, who dealt with me, bills, kept our children fed and otherwise kept a roof over my head while I was preoccupied with writing this dissertation.  She took care of everything while working full time in Santa Fe and enabling my higher education addiction.  I could never have accomplished this without her support, love, and encouragement.  She was key in keeping me on some kind of schedule to finish.  Without her gentle reminders regarding the importance of actually {\bf completing} my dissertation, I would likely be submitting this some time in 2014.

This is dedicated to you Missy.
\end{dedication}

\begin{acknowledgments}
   \vspace{1.1in}
I would like to thank my advisor, Professor Greg Heileman, for reading and re-reading my papers and dissertation, and guiding my research.  Truly widely educated with a deep understanding of an astounding number of fields, his guidance and suggestions were invaluable in helping me craft this dissertation.

Thanks to my committee for sitting through my defence, reading this dissertation, and supporting me as well as the larger computer science  and computer engineering disciplines at the University of New Mexico (UNM).  We indeed have world-class academics here at UNM, and I was so very lucky to have some of them working with me, teaching me, and helping me with my research over the (many) years.
   
I would also like to thank everyone in the Informatics Lab.  Their feedback, support, and encouragement was so very helpful in my completion of this dissertation.  Especially Pramod Jamkhedkar, my long time collaborator, and Sinan al-Saffar.  It's especially inspiring when you see someone you work with actually complete a Ph.D program.
\end{acknowledgments}

\maketitleabstract

\begin{abstract}
Information-centric networks have distinct advantages with regard to securing sensitive content as a result of their new approaches to managing data in potential future internet architectures.  These kinds of systems, because of their data-centric perspective, provide the opportunity to embed  policy-centric content management components that can address looming problems in information distribution that both companies and federal agencies are beginning to face with respect to sensitive content.  This work addresses the current state of the art in both these kinds of cross-domain systems and information-centric networking in general.  It then covers other related work, outlining why information-centric networks are more powerful with regard to usage management.  Then, it introduces a taxonomy of types of policy-centric usage managed information-centric network systems and an associated methodology for evaluating the individual taxonomic elements.  It finally delves into experimental evaluation of the various defined architectural options and presents results of comparing experimental evaluation with anticipated outcomes. This work is the end result of four conference papers (IEEE Cloud 2011, ACM DRM 2011, IEEE SoSE '11, and IEEE Cloud 2012), one invited talk (DoDIS 2012), and research performed using this system is currently submitted to two journals for publication (IJ-Closer and Dynamic Work).
\end{abstract}

\tableofcontents
\listoffigures
\listoftables

\begin{glossary}{UHR}
\item[CCN] Content-Centric Network
\item[content network] A network used to store information in which a single query to a contained content node will propagate that query to other systems until a suitable response is found.
\item[content node] A single system in a content network.
\item[content router] An node in a content network that routes queries to child nodes.  Content routers do not contain data, but just route queries.
\item[context] The environment in which a resource is used by a subject.
\item[context manager] A component that manages information associated with a context.
\item[cross domain] Actions that span two or more domains.
\item[cross domain solution] A product that facilitates interoperability between two domains.
\item[DONA] Data-Oriented Network Architecture
\item[domain (security domain)] A group of systems designed to handle certain specific types of information.
\item[domain ontology] An ontology of data from a specific domain.
\item[enclave] A subset of systems within a domain.
\item[filter] A component generally in a guard that examines a set of data and reacts specific elements of that set according to predetermined rules.
\item[guard] A system between two domains that validates information crossing from one domain to the other as appropriate for storage in the destination domain.
\item[hierarchical content network] A content network organized in a strict hierarchy.
\item[ICN] Information-Centric Network
\item[NetInf] Network of Information
\item[non-hierarchical content network] A content network that is not organized hierarchically.  An example would be a peer-to-peer system.
\item[ontology manager] A component that manages and unifies domain ontologies from different domains.
\item[ontology compiler] A component that compiles an RDF/XML ontology into Ruby.
\item[ontological hierarchy] A hierarchy of ontologies.  Generally associated with policy ontologies.
\item[overlay network] A network of nodes that communicate using the application layer of the TCP/IP or OSI network communication models.
\item[policy] A set of rules that determine how associated content can be used.
\item[policy ontology] The ontology of security policy classifiers in a specific domain.
\item[policy set] A set of policies or policy sets.
\item[PSIRP] Publish-Subscribe Internet Routing Project
\item[resource] Content that is managed from a usage perspective.
\item[rules] A specific description of how a resource can be used.
\item[smashup] A secure mashup application.
\item[subject] An actor that uses a resource.
\item[usage management] Managing all aspects of use of an item.
\item[usage management mechanism] A component that manages the use of a resource.
\item[XML] eXtensible Markup Language
\end{glossary}

\mainmatter

\chapter{Introduction}
\section{Introduction}
Current enterprise computing systems are facing a troubling future.  As things stand today, they are too expensive, unreliable, and information dissemination procedures are just too slow.

Generally, such systems still do not use current commercial resources as well as they could and use costly data partitioning schemes.  Most of these kinds of systems use some combination of systems managed in house by the enterprise itself rather than exploiting lower cost cloud-enabled services.  Furthermore, many of these systems have large maintenance loads imposed on them as a result of internal infrastructural requirements like data and database management or systems administration.  In many cases networks containing sensitive data are separated from other internal networks to enhance data security at the expense of productivity, leading to decreased working efficiencies and increased costs.

These kinds of large distributed systems suffer from a lack of stability and reliability as a direct result of their inflated provisioning and support costs.  Simply put, the large cost and effort burden of these systems precludes the ability to implement the appropriate redundancy and fault tolerance in any but the absolutely most critical systems.  Justifying the costs associated with standard reliability practices like diverse entry or geographically separated hot spares is more and more difficult to do unless forced by draconian legal policy or similarly dire business conditions.

Finally, the length of time between when a sensitive document or other type of data artifact is requested and when it can be delivered to a requester with acceptable need to view that artifact is prohibitively long.  These kinds of sensitive artifacts, usually maintained on partitioned networks or systems, require large amounts of review by specially trained reviewers prior to release to data requesters.  In cases where acquisition of this data is under hard time constraints like sudden market shifts or other unexpected conditional changes this long review time can result in consequences ranging from financial losses to loss of life.

Federal computer systems are prime examples of these kinds of problematic distributed systems, and demonstrate the difficulty inherent in implementing new technical solutions.  They, like other similar systems, need to be re-imagined to take advantage of radical market shifts in computational provisioning.

\section{Motivation}
Current policy-centric systems are being forced to move to cloud environments and incorporate much more open systems.  Some of these environments will be private or hybrid cloud systems, where private clouds are infrastructure that is completely run and operated by a single organization for use and provisioning, while hybrid clouds are combinations of private and public cloud systems.  Driven by both cost savings and efficiency requirements, this migration will result in a loss of control of computing resources by involved organizations as they attempt to exploit economies of scale and utility computing.

Robust usage management will become an even more important issue in these environments.  Federal organizations poised to benefit from this migration include agencies like the National Security Agency (NSA) and the Department of Defense (DoD), both of whom have large installed bases of compartmentalized and classified data.  The DoD realizes the scope of this effort, understanding that such technical change must incorporate effectively sharing needed data with other federal agencies, foreign governments, and international organizations \cite{proposal:info-sharing-strategy}.  Likewise, the NSA is focused on exploiting cloud-centric systems to facilitate information dissemination and sharing \cite{proposal:nsa-cloud}.

Cloud systems certainly exhibit economic incentives for use, providing cost savings and flexibility, but they also have distinct disadvantages as well.  Specifically, the are not intrinsically as private as some current systems, generally can be less secure than department-level solutions, and have the kinds of trust issues that the best of therapists cannot adequately address \cite{proposal:privacy-security-trust-cloud}.

As Pearson and Benameur \cite{proposal:privacy-security-trust-cloud} show, cloud technology is not currently as private as some organizations would like:
\begin{itemize}
\item \textit{User Data Control} --- In virtually any given Software-as-a-Service (SaaS) scenario, user data controls are sadly lacking.  Once data has been committed to a specific provider, that data is completely out of the original data owners control.  Furthermore, as we will see below, that data my not even be solely owned by the original owner anymore either.
\item \textit{Secondary Use} --- Most consumer facing social systems extensively mine user provided data for additional business advantages.  This is a common and well known secondary use for supplied data.  SaaS providers again have strong incentives to examine user provided information.
\item \textit{Offshore Development} --- Service users have no real control over who actually develops the systems a given service deploys.  Organizations have attempted to contractually limit development and support functions companies pursue to, say, the continental United States but have had very poor results with these kinds of unsupportable arrangements.
\item \textit{Data Routing} --- Both system providers and system users in fact have little control over routing issues.  Prohibiting data routing through sensitive countries is a difficult task for a single organization.
\item \textit{Secondary Storage} --- Most large-scale systems expect to use Content Delivery Networks (CDNs) to help manage content, and that expectation is heavily reflected in their physical system architectures. They simply cannot divorce use of CDNs from their systems for a single organization.
\item \textit{Bankruptcy and Data Ownership} --- Ownership and obligation to maintain expected data arrangements for a given company is not established under bankruptcy \cite{proposal:borders-info-I,proposal:borders-info-II,proposal:borders-info-III}.
\end{itemize}

Security issues also emerge from utility computing infrastructures:
\begin{itemize}
\item \textit{Data Access} --- System users have very little control over who, in the system provider's organization, is able to access their data and systems.
\item \textit{Data Deletion} --- Most savvy organizations have procedures in place to sanitize old storage elements like disk drives or backup tapes.  System users have very little control over if and how this is done when computing services are treated as a utility.
\item \textit{Backup Data Storage} --- Backup media is very difficult to encrypt, and most system providers still use tape systems as preferred media solutions for backup and storage needs.  These tapes, or copies of them, are generally stored offsite to support disaster recovery scenarios.  Security of these types of systems has been spotty to date \cite{proposal:saic-breach-I,proposal:saic-breach-II,proposal:saic-breach-III}.
\item \textit{Intercloud Standardization} --- Cloud computing systems do not have any standardized way to transfer computational units or data between systems.  Any protocols used for this kind of thing must be developed by customers themselves.  Due to the desire of providers to lock-in customers, this will likely not change as any standard development is strongly counter-incentiveized. 
\item \textit{Multi-tenancy and Side-Channels} --- Multi-tenant architectures in which multiple customers simultaneously use the same systems open those customers to covert side-channel attacks.
\item \textit{Logging and Auditing} --- Logging and auditing structures, especially for inter-cloud systems, are non-existent.
\end{itemize}

Finally, such systems suffer from internal and external trust issues:
\begin{itemize}
\item \textit{Trust Relationships} --- Trust is difficult to establish between individual cloud providers long-term.
\item \textit{Consumer Trust} --- Service users are still not entirely trusting of cloud system providers.
\end{itemize}

How to address these issues is an open research question.  Organizations ranging from cloud service providers to the military are exploring how to engineer solutions to these problems, and to more clearly understand the trade-offs required between selected system architectures \cite{proposal:assured-info-sharing}.  The problems themselves are wide ranging, appearing in a variety of different systems.  Military and other government systems are clearly impacted by these kinds of trust and security issues, and they also have clear information sensitivity problems.  This, coupled with the fact that these organizations have been dealing with these issues in one form or another for decades make them very well suited for prototypical implementation and study.

Current federal standards in place to deal with these issues in this environment are managed by the Unified Cross Domain Management Office (UCDMO).  UCDMO stakeholders range from the DoD to the NSA.  The current standard architectural model in place and governed by the UCDMO to deal with this kinds of issues are \textit{guard-centric cross domain architectures}.

\section{Confidentiality and Integrity Models}
Current models currently in active use to support information confidentiality and integrity include the Bell-LaPadula model, the Biba model,  the Clark-Wilson model, and the Brewer and Nash model.  Of these, Bell-LaPaudula and Brewer and Nash address information confidentiality, while Biba and Clark-Wilson address information integrity.

Bell-LaPadula was developed in the 1970's to address information confidentiality in the centralized, mainframe and minicomputer environments  common in military installations of the day.  It is essentially a mathematical state machine model that establishes rules with respect to how information can flow in stratified environments.  It is established around the {\it Basic Security Theorem}, which essentially states that a system with a secure initial state and only secure transitions is guaranteed to terminate in a secure final state.  It extends this theorem to establish four rules.  The first is the {\it simple security rule}.  The simple security property prohibits reading information from security levels higher than that occupied by the reader.  The next two are the {\it *-property rule} and the {\it strong *-property rule}, which prohibit writing data to any security level less than that of the writer.  The final property is the {\it ds-property}, or {\it Discretionary Security Property}.  The ds-property allows a subject to pass permissions on to other subjects at the initial subject's discretion.  This model also introduces the concepts of {\it dominance relations} and {\it tranquility}.  Dominance relations exist within partially ordered sets of classification levels, where higher classification levels are said to dominate lower levels.  This means that if a subject is cleared to access top secret material, that subject can also access secret and confidential material, as both secret and confidential material is considered to be less sensitive that top secret material.  Tranquility limits object state changes after creation.  Essentially, any object is created at a specific sensitivity level and that sensitivity level does not change with time ~\cite{Bell1973}.

The Brewer and Nash model is currently widely used in the financial services industry.  Also referred to as the Chinese Wall Model, this model changes what subjects can access based on a subjects context.  In this model, information is continually monitored so that no subject is allowed access to an object that may create a conflict of interest or other moral hazard.  More generally, access to objects is dynamically evaluated based on the context of a given subjects previous object acccesses. This enforces information confidentiality based on the context of information access ~\cite{Brewer89}.

The Biba model, developed in the late 1970's, addresses information integrity using a layered approach similar to that used by Bell-LaPadula, and was in fact developed to complement Bell-Lapadula's focus on confidentiality.  Bell-LaPadula, features a hierarchy of information sensitivity that guides access decisions.  Biba, on the other hand, is based on a hierarchy of information integrity rather than confidentiality.  Biba also has a collection of specific properties that must be adhered to.  The first, the {\it simple integrity axiom}, holds that a subject cannot access an object of a lower integrity level.  The second, the {\it *-integrity axiom}, likewise states that a subject cannot write information to higher integrity level.  The third property, the {\it invocation property}, states that subjects cannot invoke objects at higher integrity levels either ~\cite{Biba1977}.

David Clark and David Wilson presented their integrity model in 1987 to address business information integrity as Biba was thought to be more suitable for military use.  The Clark-Wilson model uses five essential components.  The first, {\it users}, are active subjects requesting access to objects.  The second, {\it transformation procedures}, are operations allowed over objects.  These general operations are any that may change the state of an object, like create, update, and delete operations, as well as simple read access.  Systems are also required to have collections of {\it integrity verification procedures} which validate the state of managed objects.  The object themselves have two distinct flavors, {\it constrained data items} and {\it unconstrained data items}.  Constrained data items are those managed via trusted procedures and monitored via integrity verification procedures.  Unconstrained items can be directly accessed.  Access and management of data items are constrained by a set of nine rules, divided into sets of {\it certification} and {\it integrity} rules.  In essence, these rules ensure that all constrained data item access is through trusted procedures, and that the trusted procedures ensure that information is maintained in a valid state.  Subjects are limited to sets of trusted procedures expressed by subject, procedure, object triples, while unconstrained data items can be accessed without limitation ~\cite{ClaWil87}.






\section{National and International Standards}
Once the Bell-LaPadula and Biba models were in developed and understood, the Department of  started another effort to more clearly categorize the security of information systems and related networks.  This effort cumulated in what is now known as the Rainbow Series of books, specifically the {\it Trusted Computer System Evaluation Criteria}, known as the Orange Book, and the {\it Trusted Network Interpretation}, or Red Book.

The Orange Book addresses security of individual computer systems.  It establishes a taxonomy of security levels at which systems can be classified.  These levels range from systems with minimal security characteristics and stretch to formally verified computer systems.  These levels establish such characteristics as separate administrator accounts, role-based security systems, data labelling, and the recognition of possible covert communications channels ~\cite{OrangeBook}.

The Red Book deals with how networks protect the confidentiality, integrity, and availability of information.  It addresses encryption, digital signatures, flow protection and confidentiality, data confidentiality, and infrastructure protection.  Like the Orange Book, it is a framework established over a set of principles to facilitate network security.  It discusses how subjects and objects need to be controlled and accounted for in computer networks just as they are within individual computer systems ~\cite{RedBook}.

Today, the Rainbow Series has largely been supplanted by the International Organization for Standardization's Common Criteria.  The Rainbow Series was seen as too rigid, and other standards too flexible and difficult to implement, leading to the development of the Common Criteria in 1993.  The Common Criteria are based on Rainbow Series, the Canadian Trusted Computer Product Evaluation Criteria (CTCPEC), and the Information Technology Security Evaluation Criteria (ITSEC).  While the Orange Book examined a system based on a Bell-LaPadula-centric perspective, the Common Criteria evaluates computing systems based on a pre-established protection profile, designed to cover a specific security requirement.  The common criteria also use a taxonomy to classify systems.  This taxonomy has seven levels, again ranging from minimally tested systems that have some assured functionality to formally verified, designed, and tested information systems ~\cite{CommonCriteria}.

Other standards commonly referenced include publications from the National Institute of Standards and Technology, particularly from their 800-series of special publications.  These publications cover everything from cloud computing security ~\cite{NIST800:144} to single computer systems \cite{NIST800:12} to web services \cite{NIST800:95}.
\section{Current Solutions}
\label{section:current-solutions}
Current federal standards in place to deal with these issues in this environment are managed by the Unified Cross Domain Management Office (UCDMO).  UCDMO stakeholders range from the DoD to the NSA.  The current standard architectural model in place and governed by the UCDMO to deal with these kinds of issues are guard-centric cross domain architectures.  As we show in section ~\ref{section:current-solutions}, the thinking behind these system architectures has remained relatively static over the past 20 years.  New thinking with regard to future internet architectures and usage management provide more powerful approaches to securing information as it flows through dynamic systems.

Current and near-future proposed solutions endorsed by the UCDMO include system architectures assembled by the NSA, Raytheon, and Booz $\mid$  Allen $\mid$  Hamilton (BAH).   The NSA has been active in this area for decades as a logical extension of their role in signals intelligence collection and processing.  Raytheon and BAH have been engaged over the past few years to provide an alternative voice and design approach to these kinds of systems, an effort met with limited success.

These cross-domain solutions are intended to enable sensitive information to easily flow both from a higher sensitivity domain to a lower sensitivity domain, and from lower to higher as well.  They generally act over both primary data (say, a document) and metadata over that primary data as well.  Note that in these system, in most cases, human intervention is still required to adequately review data prior to passing into lower security domains.

\subsection{NSA, Filtered}
The NSA conducted initial work in this area.  Their standard-setting efforts culminated in a reasonable conceptual system architecture, using groups of filters dedicated to specific delineated tasks to process sensitive information ~\cite{proposal:nsa-arch}.

\begin{figure}[!t]
\centering
\includegraphics[width=5in]{nsa-legacy-arch}
\caption{NSA Legacy Notional Architecture Model}
\label{fig:model:conceptual-model}
\end{figure}

In the scenario portrayed in Figure ~\ref{fig:model:conceptual-model}, \textit{Domain A} could very well be a private cloud managed by the U.S. Air Force, while \textit{Domain B} is a public operational network of some kind shared by coalition partners in a joint operation.

A system user attempts to send a \textit{data package} consisting of a primary document and associated metadata from \textit{Domain A} to \textit{Domain B}.  At some point, that submission reaches a \textit{guard}, which contains at least one \textit{filter chain}.  Each filter chain then contains at least one \textit{filter}.  Individual filters can execute arbitrary actions over a submitted data package and have access to any number of external resources as required.  At any point, a filter can examine the data package and reject it, at which point it will frequently wait for human review.  If a filter does not reject a data package, it passes that package onto the next filter or submits it for delivery to Domain B.

\subsection{NSA, Services}
In recent years, the NSA has extended the legacy system architecture for cross-domain information sharing to exploit service-oriented computing styles ~\cite{proposal:nsa-arch}.  Visualized in Figure ~\ref{fig:model:conceptual-model-services}, this model incorporates more modern conceptual elements and componentry.

\begin{figure}[!t]
\centering
\includegraphics[width=5in]{nsa-arch}
\caption{NSA Service-Oriented Model}
\label{fig:model:conceptual-model-services}
\end{figure}

In the view in Figure ~\ref{fig:model:conceptual-model-services}, we see on the left the \textit{Global Information Grid}, or \textit{GIG}.  On the right, we have the \textit{Distributed Service-oriented Cross Domain Solution}, or \textit{DSCDS}.  The GIG is not a truly open system --- rather, it is a loosely coupled collection of computational services handing data at a variety of levels of sensitivity, federated to provide stakeholders timely access to relevant information ~\cite{proposal:gig-arch}.  The DSCDS is essentially the embodiment of the NSA's cross-domain vision applied to service oriented computing.  This model fuses various technology choices with previous cross-domain thinking.

Indicative of this more modern system design thinking, we have a variety of services and service consumers attached to a common service bus within the GIG.  Within the DSCDS, we have groups of filters implemented as services inspecting transferred data when moved over the bus.  Finally, all of this interaction is managed by a management interface and controlled by an orchestration engine accessing a centralized group of policies.

Note that here we have begun to access a common policy repository for various types of security metadata regarding primary data elements.

\subsection{Raytheon}
In the past few years, Raytheon has offered a new model for cross domain use influenced by the NSA service-oriented model ~\cite{proposal:raytheon-arch}.

\begin{figure}[!t]
\centering
\includegraphics[width=5in]{raytheon-arch}
\caption{Ratheon Model}
\label{fig:model:conceptual-model-ray}
\end{figure}

The model in Figure ~\ref{fig:model:conceptual-model-ray}   is more grounded in the actual technical environment this kind of solution would be embedded within.  Here, we have the Non-secure Internet Protocol Router Network (NIPRNet) as one domain, and the Secret Internet Protocol Router Network (SIPRNet) as the other.  Here, NIPRNet is the lower security domain (lowside), and SIPRNet the higher security domain (highside).  This particular view shows the motion of data from the high side (SIPRNet) to the low side (NIPRNet).

Here, a data request is submitted from SIPRNet first two the \textit{XML Security Gateway} which calls into the \textit{Orchestration Engine} for policy validation.  The Orchstration Engine then coordinates calls into a \textit{Policy Repository} as well as to a collection of external \textit{Support Services}.  Once rectified against these elements, the request is passed into the \textit{Cross Domain Guard} which routes the request into the \textit{Unclassified Enclave} in NIPRNet.  Here, the request is passed directly through the lowside \textit{XML Security Gateway}, without rectification, onto the \textit{Service Provider}.  The response from the Service Provider is then passed back to the requester via the inverse path.

This model also begins to use a centralized policy repository, just as the NSA Service Model.  It also uses a single cross domain guard to transfer information from both the highside to the lowside, and vice-versa.

\subsection{Booz $\mid$ Allen $\mid$ Hamilton}
BAH submitted a competing model, also in 2009 ~\cite{proposal:bah-arch}.  In fact, both Raytheon and BAH presented their models under competitive contract to the UCDMO at the same conference, so the domain application is not coincidental.

\begin{figure}[!t]
\centering
\includegraphics[width=5in]{bah-arch}
\caption{Booz $\mid$ Allen $\mid$ Hamilton Model}
\label{fig:model:conceptual-model-bah}
\end{figure}

Figure ~\ref{fig:model:conceptual-model-bah} embodies BAH's thinking with respect to cross domain information management.  We have a \textit{Domain A} as a high security domain, and \textit{Domain B} as a low security domain.  Here, we again have dataflow from the highside to the lowside through the cross domain management system.

While not as detailed as the Raytheon proposal, this does have similar elements.  Here, we data first travels from Domain A into the \textit{Interface Segment for Domain A}, similar to the secret enclave used in the Raytheon model.  From there, it moves into the \textit{CI Segment}, which in turn submits the transferring data into the \textit{Filter Segment}.  From there, the package is moved into the \textit{Interface Segment for Domain B}, and then onto \textit{Domain B}.  The \textit{Administrative Segment} provides managment and oversight of the system as a whole.

Note the absence of specific policy-centric elements.  This system is reliant on specific policy-agnostic content filters as well.
These kinds of cross-domain solutions still have clear similarities, and in fact have not progressed far beyond the initial notions of how these kinds of systems should work.  They still, for example, all use some kind of filter chaining mechanism to evaluate whether a given data item can be moved from a classified to an unclassified network.  Both NSA models used filters explicitly, as did the BAH model.  They all use a single guard as well, a sole point of security and enforcement, providing perimeter data security, but nothing else.  In each of these current system architectures, users are only allowed to exchange one type of information per domain.  The physical instantiations of these models are locked by operational policy to a single classification level.  Users cannot, for example, have Top Secret material on a network accredited for Secret material.  Finally, these models violate end-to-end principles in large service network design, centralizing intelligence rather than pushing that intelligence down to the ends of the system ~\cite{Clark:1995:DPD:205447.205458}.

End-to-end principles are generally considered core to the development of extreme scale, distributed systems.  Essentially, one of the key design decisions with respect to the early internet was to move any significant processing to system end nodes, keeping the core of the network fast and simple.  Known as the end-to-end principles, this design has served the internet well, allowing it to scale to sizes unconceived when originally built.  Current cross domain systems are placed at key routing points between sensitive networks.  These locations are core to information transfer between systems and as a result violate the initial design principles upon which the internet was founded.  There does exist some belief that end-to-end principles need to be modified to support future networks, but nevertheless, current cross domain systems still violate the basic ideas behind large, scalable networks by placing complex application-specific logic directly and only in the core of a given sensitive network ~\cite{Blumenthal:2001:RDI:383034.383037}.

Future systems will generally demonstrate decentralized policy management capabilities, infrastructural reuse, the ability to integrate with cloud systems, and security in depth.  Policy management will need to be decentralized and integrated within the fabric of the system.  The system is both more secure and resilient as a result, better able to control information and operate under stressful conditions.  Multi-tenancy can lower costs and increase reliability and is furthermore a common attribute of cloud systems.  An appropriately secured system facilitates integration of computing resources into multi-tenant environments.  The ability to handle multi-tenant environments and to reliably secure both data at rest and data in motion leads to computational environments deployable in cloud systems.  Finally, systems must operate under \textit{all} conditions, including when they are under attack or compromise ~\cite{proposal:ron-ross} and provide protection to sensitive data in depth.

\section{Information-centric Networking}
\label{section:information-centric-networking}

%who what why how % when where
%
%I. What is it
%	Define it	
%	who they are
%	why are they important
%	what problem are they solving
%	introduce various different approaches
%		differentiate them

Information-centric networking is a new approach to internet-scale networks.  In general, it takes extensive advantage of data locality, cache data aggressively, decouple information providers from consumers, and use a content-centric perspective in network design. The overriding goals of this approach include providing higher information availability through better network resilience and implementing systems that more closely reflect today's use, focusing on heterogeneous systems from mobile to static access \cite{6231276}.  Four of the leading projects implementing these ideas are Data-Oriented Network Architecture (DONA) ~\cite{Koponen:2007:DNA:1282427.1282402}, Content-centric Networking (CCN) ~\cite{NDN,CCNx}, the Publish-Subscribe Internet Routing Project (PSIRP) ~\cite{PSIRP}, and the Network for Information (NetInf) ~\cite{NetInf}.

%		
%II. Common Features
%	why they have common features
%	Primitives and design approaches
%		contrast with past internet paradigms
%		current API structural commonalities (use IDL!)
%		how they would work
%		
%III. Differentiating Features
%	what are they
%	why are the different
%	what capabilities to the differences give us
%
%IV. Current Projects (just text, paragraph per, ref projects and survey paper)
%
%V. Conclude
%	impact if successful
%	what else? motivate importance of this work!
	
\section{Other Related Work}
This work introduces the notion of usage management embedded in a delivery network itself.  It also provides an in-depth analysis of the challenges and principles involved in the design of an open, inter-operable usage management framework that operates over this kind of environment. Besides referencing the material covered in depth to portray the current state of the art, the analysis includes application of well-known principles of system design and standards~~\cite{Blumenthal:2001:RDI:383034.383037,Clark:1995:DPD:205447.205458,ClWrSoBr:02}, research developments in the areas of usage control~~\cite{PaSa:04,JaHeLa:10}, policy languages design principles~~\cite{JaHeMa:06}, digital rights management (DRM) systems~~\cite{JaHe:09},  and interoperability~~\cite{JaHe:04,HeJa:05,KoLaMaMi:04,coral,marlin} towards the development of supporting frameworks.

While a large body of work exists on how overlay networks can use policies for \textit{network} management, very little work has been done on using usage policies for \textit{content} management.  The primary contribution in this area focuses on dividing a given system into specific \textit{security domains} which are governed by individual policies ~\cite{4457175}.  This system fits into this proposed taxonomy as an $\alpha$-type system as it has domains with single separating guards.

A large body of work currently exists with respect to security in and over overlay networks.  These kinds of techniques and this area of study is vital to the production development and delivery of overlay systems, but is outside the scope of this work.

\chapter{Usage Management in Information-centric Networks}
\label{section:capabilities}
\color{red}
Need more detail here expanding and establishing claims.  Examples on why layer breaking is bad.  Policy size being too large for single packets, and why that is bad.  Outline why re-assembling large documents for detailed analysis is hard, and needed when information spans packets.  Add cites.
\color{black}

Usage management is better served by information-centric networks than traditional packetized networks.  The basic structure of packet networks facilitates simple and efficient data transfer, but is fundamentally based on certain design assumptions that render network-centric usage management difficult at best and impossible at worst ~\cite{Cl:88,SaReCl:84}.  Information-centric networks, taking a very different approach to network design, are much more amenable to embedded content control based on their different design principles ~\ref{section:information-centric-networking}.

Current packet-based systems share two common design principles.  Strict layering, in which upper layers only use services that exist in lower layers which in turn have no knowledge of upper layers, and limited runtime packet sizes.  In internet systems, switching and routing occur in the lower layers of the OSI model (layers two and three) ~\cite{TODO}.  These decisions are made based on a priori knowledge of a given network topology, and are not impacted by transmitted content ~\cite{TODO}.  In fact, access to application content occurs at much higher levels (specifically, layer seven).  As a result of strict service layering, the content information needed to make content-sensitive routing decisions is simply not available without breaking layer encapsulation.  Likewise, policies associated with content can be arbitrarily large ~\cite{TODO}.  As a result, they can exceed maximum packet sizes defined in packetized networks.  Furthermore, as content-sensitive networks must evaluate defined policies prior to routing content, any policy to be evaluated must be completely downloaded into a router and analyzed for suitability for transmission prior to any packet routing, leading to inevitable bottlenecks as content is queued behind the policy elements.

Information-centric neworks are based on different primitives ~\ref{section:information-centric-networking}.  Specifically, they are based on named data objects with strict name-data integrity, as well as other associated principles.  This different abstraction makes policy evaluation and content binding simpler, as content can be bound either in-line to policy or via specific naming conventions.  In these systems, once content is located by name, it is returned to the requester either via a predefined path mirroring the original request path or a variable response path.  In either case however, all content and associated policy is available at each routing node in that return path, and can be evaluated for suitability of transmission.

As a result of these fundamentaly different underlying models, information-centric networks in the next-generation internet enable usage management capabilities that are very problematic to implement and enforce in current internet architectures without dedicated application-level overlays.
\section{Characteristics}
\label{section:characteristics}
Examining content at each network node or router point can certainly impact performance and by extension availability.  Furthermore, we need to establish that this kind of dynamic dispatch will actually guarantee delivery along the most secure path.  We can help to increase the throughput of the network overall by taking advantage of the volatility characteristics of the network.  With respect to delivery, we need to show that by selecting optimum paths at given network points the overall selected path will have the appropriate security characteristics outlined by any policy associated with delivered content.

To begin with, we propose that in a given aggregate path between two points if we make local decisions with respect to routing based on specific security criteria at interleaving points the path as a whole will adhere to those security criteria.  Essentially, this implies that we can use a greedy algorithm with respect to security and routing and that the algorithm will yield an optimal security path.  It is important to recognize that this is key to establishing a secure route between two specific points.  Furthermore, in a given route, that route must be viewed temporally as well, in that each link may not be optimal when the delivered data element reaches a destination, but each link was optimal at the time it was selected, and by extension, when the aggregate path is reviewed, it would likewise be optimal with respect to time of traversal.  Finally, local nodes may very well have knowledge about the local environment that cannot be known by a centralized routing authority.  Allowing local routing decisions with respect to security can help take advantage of this locality.



\color{red}
PROOF HERE
\color{black}

We likewise would like to be able to establish certain optimizations with respect to known system state.  Specifically, when we look at a given edge leading away from a node in a path, that edge has certain information associated with it that facilitates routing decisions.  These parameters could be the known state of the network, beliefs with respect to potential route compromise, or a known state of cyber-attack.  These parameters could effect an entire network or just a single edge between two nodes.

\color{red}
I. What are the areas of concern? routing (via greedy alg.) and efficiency (via dynamic programming)

II. How do we know greedy alg. wrt security will work?
A. why is this important?
B. how would it be applied?
C. Outline proof
1. proof


III. How do we know dynamic programming wrt efficency will work? what are the limits?

IV. Wrap up?


What kind of characteristics do we care about? 

more rigor over layers and anlysis here.  Rigorous proof of capabilities wrt content evaluation? prove overlapping subproblems and optimal substructure, and constraints thereof.
\color{black}
\section{Taxonomies of Usage Management Overlay}
A clear taxonomic organization of potential steps in approaching finer grained policy based usage management helps in describing the difficulties inherent in developing potential solutions as well as aiding in planning system evolution over time. Here, we have five distinct types of integrated policy-centric usage management systems, as shown in Table \ref{table:model:taxonomy}.  Of these five, only the first two levels are represented in current system model.

\begin{table}[tp] %
\centering %
\begin{tabular}{clcc}
\toprule %
$ Name$ 	& $Description$ \\\toprule %
$\phi$ 		& The initial level of this taxonomy, $\phi$ classified systems \\
 			& have a single guard without policy-based control \\\midrule
$\alpha$	& $\alpha$ classified systems have a single guard by have begun \\
			& to integrate policy-based control \\\midrule
$\beta$		& Systems that have begun to integrate policy-based control with \\
			& router elements are in the $\beta$ category \\\midrule
$\gamma$	& Systems that have integrated policy-based control with routing \\
			& and computational elements \\\midrule
$\delta$	& Continuous policy-based control with \textit{smart licensed} artifacts \\\bottomrule
\end{tabular}
\caption{Proposed Usage Management Taxonomy}
\label{table:model:taxonomy}
\end{table}

In this taxonomy, it is not required that systems pass through lower levels to reach higher ones.  This taxonomy represents a continuum of integration of usage management controls.  Systems can very well be designed to fit into higher taxonomic categories without addressing lower categories.  That said however, many of the supporting infrastructural services, like identification management or logging and tracing systems, are common between multiple levels.

The taxonomy itself starts with the current state, integrating policy evaluation systems into the network fabric gradually, moving away from filters, then by adding policy evaluation into the routing fabric, then the computational nodes, and finally by incorporating evaluation directly into content.

\subsection{$\phi$-level Overlay Systems}

\subsection{$\alpha$-level Overlay Systems}

\subsection{$\beta$-level Overlay Systems}

\subsection{$\gamma$-level Overlay Systems}
\section{Analysis}
\color{red}
What do we expect to see when we inject UM into ICN? what are the tradeoffs? what kind of conclusions can we extract?
\color{black}

\chapter{Experimental Configuration}
% \label{section:experiments}
\color{red}
outline experiments here, what we are measuring, how we will measure, etc. how are we going to prove our assertions from the previous chapter?
\color{black}
\begin{figure*}[!t]
\centering
\includegraphics[width=6in]{cross-domain-prototype}
\caption{Simulation Logical Configuration}
\label{fig:model:cross-domain-prototype}
\end{figure*}

\section{Overlay Implementation Concerns}
A key concept in our current work is the separation of content management from physical communication networks.  In the past, content was controlled via partitioning and physical network access management.  Physical networks were tightly controlled as a way to manage access to sensitive content.  Classified networks in common use today are canonical examples of this kind of approach to content management.  Access to these networks is tightly controlled by classification authorities and the ability to transfer content from these networks to more open systems is rigorously managed.  Corporate systems have also commonly used this kind of approach, though not usually with so much regulation or rigor.

This kind of approach is not scalable however.  It imposes huge costs and infrastructural requirements that are becoming too large to effectively manage.  Furthermore, future systems containing sensitive information require similar security features, and simply cannot be developed without custom controlled infrastructure.  Health care systems, for example, have huge security needs and a more finely grained level of application than even deployed government systems.  These systems will contain exabytes of data, all of which needs to be explicitly controlled, managed, and reviewed by those associated with specific managed records.

Separating content networks from physical networks enables network infrastructure virtualization and multi-tenancy.  Use the popular file-sharing system BitTorrent as an example.  BitTorrent is a content network optimized for download efficiency.  It run over traditional TCP/IP networks, but manages traffic according to specialized algorithms unique to BitTorrent.  These algorithms take advantage of the asymmetry between upload and download speeds of typical home-use Internet systems in which upload speeds are regularly an order of magnitude slower than download speeds.  By partitioning content into distinct sections and downloading them from multiple clients, a downloading node can effectively use all available download bandwidth and is no longer necessarily constrained by the upload bandwidth of a serving peer system.  We use a similar approach, in that our hypothesized systems also overlay TCP/IP traffic, but rather than optimizing download speeds we focus on content usage management.

Just as systems like BitTorrent runs over current established protocols, usage management overlay systems could as well.  They support multi-tenant cloud computing systems by providing secure compartmentalized access to managed information.  They also support the ability to create and use integrated overlay systems between multiple cloud providers, supporting running of overlay components in systems hosted at Amazon while accessing nodes executing on Rackspace infrastructure.

Content networks must deal with situations analogous to those encountered in previous physical systems.  Specific examples include cross-domain monitoring and content mashing.  Both problems are currently areas of active research within physical networks and need extensive examination in overlay systems as well.

To begin with, in content-specific overlay networks, cross-domain routing can become an even more pervasive issue.  Currently, cross-domain data processing guards are installed on the perimeter of sensitive networks where they can monitor and manage outgoing and incoming traffic.  In content networks, these kinds of systems can begin to multiply within the information transmission fabric.  In physical networks, the network topology is fixed and is established when the network is installed.  After installation, changes in the essential network topology are cost-prohibitive and correspondingly rare.  Overlay systems do not suffer from this high cost of change, and can easily morph from one topology to another.  As additional content enclaves appear within a given overlay topology, the need for content usage management between those enclaves increases.

Mashup scenarios become similarly common.  As additional sources of accessible data appear, opportunities for inappropriate data combinations increase at best geometrically.  Data combinations need to be likewise managed to prevent inappropriate data combinations.

\section{Initial Prototype Implementation}
Our first completed prototype shows that overlay routers can in fact use licenses bundled alongside content to modify transmitted content based on dynamic network conditions.  Running on a single host over HTTP, it simulates two content domains and communication between them.  The communication link has uncertain security state and changes over time.  Note that this prototype currently runs on a single host with varying ports, but it could easily run on multiple hosts as well.  The current single host configuration is simply to simplify system startup and shutdown.

License bundles are hosted on the filesystem, though they could be hosted in any other data store.  These artifacts are currently XML.  They are stored in a directory, and the license file has a LIC extention while the content file has an XML extension.  Both the content and the license files have the name of the directory in which they reside (for example, if the directory is named test, the license file is named test.lic and the content file test.xml).  In this context, the directory is the content bundle.  The license and content files are simply documents and port to document-centric storage systems like MongoDB easily.  They can certainly be stored in traditional relational databases as well.

The system itself has two domains, Domain 0 and Domain 1.  Each domain consists of a client node and a content router node.  Requests are initially served to client nodes.  If client nodes do not contain the requested content, they the forward that request to their affiliated content router.  The content router will send that request to all the content routers of which it is aware.  Those other routers will then query associated client nodes for content.  If the requested content is in fact found, it will be returned to the original requesting router and then to the requesting node.  If the content is not found, HTTP status 404 codes are returned to requesting routers and nodes.

All router-to-router content traffic is modified based on security conditions.  A Context Manager maintains metadata regarding network paths.  If a given network path is only cleared for data of a certain sensitivity level, a transmitting router will remove all license information and content that is associated with higher sensitivities, and then transmit only information at an appropriate sensitivity level over the link.

Figure \ref{fig:model:cross-domain-prototype} shows the prototypical workflow through the system across the domains, and Figure \ref{fig:model:prototype-physical-config} shows the current system configuration of the simulation, with the cross-domain link highlighted in red.  The system is current configured to use ports 4567 through 4571.

All content requests are via HTTP GET.  Link status can be changed via HTTP POST and we use the CURL command to exercise the network.

This proof-of-concept does implement a simple overlay network for usage managed content over HTTP, easily extensible to HTTPS.  Changes in the context of the network dynamically change the format of transmitted content.  All source code for this simulation is publically available on GitHub, at https://github.com/cclamb/overlay-network, with documentation on how to run the simulation.

\begin{figure}[!t]
\centering
\includegraphics[width=3in]{prototype-physical-config}
\caption{Physical Simulation Configuration}
\label{fig:model:prototype-physical-config}
\end{figure}

\section{Initial Prototype Results}

Policy and content delivery over HTTP possible

Ruby/Sinatra will support HTTP overlay development, CURL for usage

Information Filtering Expectations

Extension into larger distributed system feasible

\section{Inter-Provider Cloud Configuration}
Over the month of June we established our initial technical baseline for upcoming content network development.  We have created and deployed baseline system images in both Amazon's Elastic Compute Cloud (EC2) and Rackspace Servers infrastructures.  We have also created and exercised our deployment, configuration, and logging systems to enable distributed monitoring and centralized reporting.  Overall, we currently have 20 nodes running with two distinct providers geographically dispersed across the continental United States.  This leads to a distinct requirement for a centralized system with distributed access for both initial configuration information as well as logging and auditing.  We have implemented this required infrastructure using Amazon's Simple Storage Service (S3), accessible from both Rackspace and Amazon hosted virtual machines.
The specific technical components are Amazon EC2, Amazon S2, Rackspace Servers, and GitHub.  Both EC2 and Rackspace nodes are Ubuntu virtual machines, albeit at different versions, as we run Ubuntu version 11.04 in Rackspace and Ubuntu Version 12.04 in Amazon's infrastructures.  These systems are provisioned with Git, Ruby, the Ruby Version Manager (RVM), and supporting libraries.  They all run as micro-intances or equivalent, and are bootstrapped with the appopriate project information to begin to participate as an overlay network node.  While EC2 and Rackspace Server infrastructures are infrastructure-as-a-cloud (IaaS) offerings supporting virtual machine instances of various types, Amazon S3 is a simple key-value store.  Running with REST sematics over HTTP, S3 stores arbitrary documents associated with specific keys in buckets.  These documents can be downloaded by any authorized participant, where authoriztion state is proven by possention of a secret key.  In this way, we can store the global configuration of a specific overlay network in a single location from which every node can access informationm with respect to their pending role and needed configuraiton information.  Likewise, all overlay network state can also be saved to centralized buckets for later analysis.  Finally, Github is a centralized source code repository used to share code between all participating nodes.  Prior to each content network instantiation, each node checks the repository for updates, and downloads them if they exist.

All data saved within S3 is serialized in a text-based data serialization language known as YAML.  YAML is a widely supported hierarchical data representation language with support within the Ruby core platform.  This enables us easily serialize Ruby-native data structures to text-based representations for storage within S3.  More importantly, it simplifies post-experimental data analysis as any information logged to the centralized logging system during a given experimental run can be easily read and analyzed after the fact.

In order to manage and initialize all overlay nodes, we use Capistrano.  Capistrano is a distributed deployment system initially used to manage large clusters of Ruby-on-Rails systems.  It has since expanded into a general-purpose distributed deployment toolchain, tightly integratd with Git.  This allows us to bootstrap different configurations of networks from a single command-and-control node simply and efficiently.

\begin{table*}[tp] %
\centering %
\begin{tabular}{clcc}
\toprule %
$Category$ 				& $Components$ 								\\\toprule %
$Infrastructure$ 		& Amazon S3, Amazon EC2, Rackspace Servers 	\\\midrule
$Operating Systems$		& Ubuntu 11.04, Ubuntu 12.04 				\\\midrule
$Technologies$			& Ruby (Sinatra, Capistrano, YAML) 			\\\midrule
$Supporting Systems$	& Git, Github 								\\\bottomrule
\end{tabular}
\caption{Supporting Components}
\label{table:model:components}
\end{table*}

All these infrastructural elements, protocols, and technology components have been successfully tested, allowing for unified control and configuration of large, distributed overlay systems.  We have successfully tested our logging systems, and integrated them with the Ruby runtime for ease of access.  We have also passed configuration information to both Rackspace and Amazon EC2 systems and verified access from all participating nodes.  Finally, we have successfully exercised the ability to dynamically update all participating nodes from Github as well as the capability to manage the system via Capistrano.



\chapter{Experimental Results and Conclusions}
Experiments using this inter-cloud framework yield promising support for this approach.  They show only a slight degradation of information availability as a result of this network permeated security approach, with redaction and encryption demonstrating the smallest degradation at a higher impact on delivered information integrity.  Rerouting-based approaches have the most performance degradation. Encryption generally has the smallest impact on information integrity.  This is most evident when network effects are removed from evaluation.  Non-hierarchical and hierarchical networks have very similar go performance with respect to content availability as well.

The goal of this experimental work was to characterize confidentiality, integrity, and availability impacts of these information-centric network security approaches in both hierarchical and non-hierarchical configurations.  The specific strategies addressed were redaction, rerouting, and protection (via encryption), and these strategies were evaluated from the perspective of confidentiality, integrity, and availability over hierarchical and non-hierarchical networks, and on standalone nodes. Confidentiality was measured via the control used to protect information.  Removing information entirely provided the highest measure of protection but is akin to unplugging a computer to improve its cyber-security posture.   Routing information through a more secure channel is the next most powerful approach, followed by sensitive information protection via strong encryption.  A 256-bit AES-CBC encryption scheme was used in this work.  Availability was measured by the delivery of information and the time required to ensure information delivery, measured by end-to-end network performance.  Integrity is a function of the alterations to the information required for secure delivery in the tested scenario.  Unaltered information has the highest integrity, followed by information that is still complete but protected via encryption, information that has been divided and rerouted, and finally information that has had content redacted.  Though combinations of strategies in a given network can be specified, as strategies are specified by network node, in these experiments only a single strategy in each network was used to more clearly attribute strategy performance impacts. Identical policies were used in each simulation to ensure the same amount of required usage management actions, limiting the effects on availability to the approach rather than differing policy.  In each case, a control simulation that did not incorporate any usage management was run to provide a performance baseline.  

\section{Hierarchical Networks}
In these tests, a simulated $\gamma$-categorized system was examined.  This is the kind of system that organizations like the UCDgitMO have identified as the final goal state of their work, systems that incorporate policy-centric management in the fabric of systems and networks (12).  The kind of components required to do this kind of policy-based content-sensitive evaluation do not currently exist, and components of these kinds of systems are only now beginning to emerge.  Systems like OpenFlow, when they have stronger hardware support, can begin to provide some of these kinds of capabilities.  OpenFlow enabled systems are not yet common or widely used however, and though they do provide the needed control for these kinds of systems, the do not supply the necessary policy interpretation and evaluation.  As a result, this experimental work was conducted over an HTTP overlay network, at the application layer.  Using a document-focused protocol makes content evaluation simpler as well, as systems can evaluate all content when it transits a network rather than maintaining a buffer of content required when processing packet-level communications.

In order to develop a stronger perspective on the network performance, delivery times were measured from three separate nodes.   One node is hosted in Comcast's infrastructure (a large local internet service provider), one at Amazon, and another at Rackspace.  The tested network had four levels.  The first level had a single router node.  The next level had two routers, both connected to the router in the first level.  The third level contained four routers, two attached to each of the routers at the level just above.  Finally, the fourth level contained nodes, distributed so that two level three routers had three nodes, one level three router had two nodes, and the last level three router had four nodes.  The first three levels were essentially a binary tree.  The network was queried from five different locations.  The node that contains the content was queried directly (the home node).  A node under the same router as the home node was then queried for content (the peer node).  Next, queries were sent to a node under a different router, but connected to the same second level router (the neighbor node).  Finally, two nodes on the other side of the network were queried for content (the distant (1) and (2) nodes).  Each node was queried for content 50 times in each strategy, for a total of 200 queries per node.  Each figure is the result of 1000 individual sample measurements collected throughout the day and throughout the week.

%\begin{figure}[!t]
%\centering
%\includegraphics[width=6in]{strategy_effects_az}
%\caption{Hierarchical Results from Amazon}
%\label{fig:model:amazon-results}
%\end{figure}

\begin{figure}[htbp]
\begin{minipage}[b]{0.5\linewidth}
\centering
\subfigure[Mean Response Times]{\includegraphics[width=\linewidth]{strategy_effects_az}}
\end{minipage}
\begin{minipage}[b]{0.5\linewidth}
\centering
\subfigure[Standard Deviation of Response Times]{\includegraphics[width=\linewidth]{strategy_effects_stdev_az}}
\end{minipage}
\caption{Hierarchical Results from Amazon}
\label{fig:model:amazon-results}
\end{figure}

Figure ~\ref{fig:model:amazon-results} shows performance results from the Amazon testing node.  The access times for the content from the home, peer, and neighbor nodes were by far the smallest.  As the testing node was hosted in the same datacenter as these three nodes, that was to be expected.  The access times for both distant nodes was, however, surprisingly high.  With that in mind, the overall trend for response times is sensible however, with access time increasing as the requesting node is farther away from the content in the information network.  Queries from distant nodes need to traverse five information routers, while home, peer, and neighbor nodes only traverse one, two and three, respectively.  Also surprising was the finding that rerouting was generally more expensive from an availability perspective than encryption-based approaches.  This is likely attributable to the costs associated with attaching to the external SMTP server, hosted at Google, used as the out-of-band communications channel.  Also evident is remarkable performance variability.  Specifically, results from the Distant (1) node show remarkable variance.  Variance also seems to increase in this group of samples as requests move farther away from the home node.  Control data was collected at different times than experimental data, and infrastructural demands seem to have driven the control data availability to be less than that of other, managed approaches.  Overall, this evidence of variable performance due to external provider demands leads to the conclusion that overall, the availability costs of the various approaches are in fact negligible.

%\begin{figure}[!t]
%\centering
%\includegraphics[width=6in]{strategy_effects_rs}
%\caption{Hierarchical Results from Rackspace}
%\label{fig:model:rackspace-results}
%\end{figure}

\begin{figure}[htbp]
\begin{minipage}[b]{0.5\linewidth}
\centering
\subfigure[Mean Response Times]{\includegraphics[width=\linewidth]{strategy_effects_rs}}
\end{minipage}
\begin{minipage}[b]{0.5\linewidth}
\centering
\subfigure[Standard Deviation of Response Times]{\includegraphics[width=\linewidth]{strategy_effects_stdev_rs}}
\end{minipage}
\caption{Hierarchical Results from Rackspace}
\label{fig:model:rackspace-results}
\end{figure}

Figure ~\ref{fig:model:rackspace-results} shows similar results to Figure ~\ref{fig:model:amazon-results}.  Here, the query times are much higher for the home and peer nodes, but actually lower for the distant nodes.  In this case, the content is still hosted in Amazon's infrastructure, but the testing node is at Rackspace.  As a result, the longer response time for content from the home node is to be expected.  Queries to distant nodes are actually shorter than the previous calls into distant nodes from Amazon.  This stems from the fact that the distant nodes are both hosted at Rackspace.  This locality shortens the round trip distance for a request.  Previously, from Amazon, a content request would need to travel from Amazon's east coast data centers to the Rackspace data center in Dallas, then back to the east coast for content, then back to Dallas, then back to the east coast.  In this test, the request only travels from Dallas to the east cost, and back.  Nevertheless, the overall performance profile is sensible, reflecting the expected shorter latency between home, peer, and neighbor nodes when compared to distant nodes.  Similar to amazon, cases when the control latency is higher than experimental latency emerge, indicating some amount of infrastructure performance variability.  In Figure ~\ref{fig:model:rackspace-results} however, it is evident that overall encryption and rerouting impact performance more than redacting, as would be expected.  Rerouting again has high overall impact, likely as a result of contacting Google's remote SMTP services.

Rackspace results also seem to exhibit less variance than equivalent results when testing from Amazon, as shown by the standard deviations in Figures ~\ref{fig:model:amazon-results} and ~\ref{fig:model:rackspace-results}.  Rackspace results are dominated by routing and encryption variance as well, indicating variance as a result of system node performance rather than network effects.  Network effects would touch on all results more uniformly, as shown in Figure ~\ref{fig:model:amazon-results}.  Processing induced variance would preferentially effect those strategies that require more processing, like encryption and rerouting.  This is the pattern seen in Rackspace testing.

Figure ~\ref{fig:model:comcast-results} Shows performance results measured from Comcast.  Interestingly, they show significant variability when accessing nodes hosted at Amazon, and more predictable performance when accessing nodes in Rackspace's infrastructure.  The overall variability does not follow the expected pattern of shorter response times when accessing content from nodes close to that content, except in a few cases.  This illustrates the kind of performance variability one can expect from an external service provider.  Interestingly, the variance of performance in Comcast testing is somewhat low, except for a single spike in the control case resulting from a single unusually slow response:q:q!
.

Integrity impacts are the result of approach rather than platform.  Redacting content destroys information integrity, as information is removed and not delivered to requesters.  Encryption maintains integrity the best of the three alternatives as information, even though encrypted, is still delivered, and delivered in the context of the query response at that.  Rerouting is better than redaction, in that sensitive information is still delivered, but worse than encryption, as it is not delivered within the response context and is sent out-of-band. Simulations removed sensitive information from the information network and dispatched it to a user's email address via SMTP over TLS when the selected strategy was rerouting.  This impacts information availability, as email delivery times can be highly variable.  In these experiments, delivery could take anything from a few seconds to a few minutes.

%\begin{figure}[!t]
%\centering
%\includegraphics[width=6in]{strategy_effects_local}
%\caption{Hierarchical Results from Comcast}
%\label{fig:model:comcast-results}
%\end{figure}

\begin{figure}[htbp]
\begin{minipage}[b]{0.5\linewidth}
\centering
\subfigure[Mean Response Times]{\includegraphics[width=\linewidth]{strategy_effects_local}}
\end{minipage}
\begin{minipage}[b]{0.5\linewidth}
\centering
\subfigure[Standard Deviation of Response Times]{\includegraphics[width=\linewidth]{strategy_effects_stdev_local}}
\end{minipage}
\caption{Hierarchical Results from Comcast}
\label{fig:model:comcast-results}
\end{figure}

Confidentiality is likewise impacted primarily by approach and not by infrastructure.  Redacting sensitive content provides the best confidentiality protection, as sensitive content is simply not exposed.  Encryption is likely the worst solution from a confidentiality perspective as content encryption is a delaying tactic against a determined, well-resourced adversary.  Rerouting may be better or worse than encryption as an approach, depending on the confidentiality of the out-of-band channel.  If the security of that channel can be guaranteed, then it is likely a better approach.  If, on the other hand, the security of that channel is more variable or difficult to ascertain, encryption may be a more reliable approach.

Overall, results show that, from a performance perspective, the rerouting approach fares the worst, but only slightly, and certainly not in all cases.  Both results from Amazon and Rackspace, in Figures ~\ref{fig:model:amazon-results} and ~\ref{fig:model:rackspace-results}, show encryption as generally taking the second largest performance hit, just following rerouting.  Furthermore, network effects have a much larger impact on performance than information protection approaches.  The query to the home node is an excellent predictor of overall network stability, as content delivered directly from a home node is only subjected to the selected information protection strategy once.  Note that when queried from Amazon or Rackspace, the home node timing results are very close to uniform.  Queries from Comcast, however, are much more varied, indicating more highly variable quality of service within the Comcast network.  This is also supported by the gross distribution of response times.  Within both the Amazon and Rackspace networks, the farther a queried node is from the content requested, the worse the latency, as expected.  Comcast's network has a much more uniform information network response time overall as the processing time of the information network simulation is overshadowed by the highly varied performance of Comcast's physical network.  Availability is surprisingly uniform across all confidentiality strategies, showing little impact on end-to-end processing times.

\section{Non-Hierarchical Networks}
In order to test non-hierarchical networks, a simple branching network of participants was used, identical in form to the hierarchical network, though queries could be routed through the network from any point.  Queries could come into any node on the network, and would propagate through the network to the requested content, evaluating the returned content as it passes back through the network in response to the initial query.

In these experiments, the node that contains the content was queried, then the node immediately next to that content node, and so on, to a distance of five nodes.  The home node again contains content, and the additional nodes are marked by the distance in node count from the home node, starting with Neighbor (1), proceeding through Neighbor (5).  The non-hierarchical network was queried from Rackspace, Amazon, and Comcast, for a total of 200 queries per individual node, testing the system 50 times per each confidentiality strategy.  Here, each figure is the result of 1200 individual sample measurements, again collected throughout the day and throughout the week.

%\begin{figure}[!t]
%\centering
%\includegraphics[width=6in]{nh_strategy_effects_az}
%\caption{Non-Hierarchical Results from Amazon}
%\label{fig:model:nh-amazon-results}
%\end{figure}

\begin{figure}[htbp]
\begin{minipage}[b]{0.5\linewidth}
\centering
\subfigure[Mean Response Times]{\includegraphics[width=\linewidth]{nh_strategy_effects_az}}
\end{minipage}
\begin{minipage}[b]{0.5\linewidth}
\centering
\subfigure[Standard Deviation of Response Times]{\includegraphics[width=\linewidth]{nh_strategy_effects_stdev_az}}
\end{minipage}
\caption{Non-Hierarchical Results from Amazon}
\label{fig:model:nh-amazon-results}
\end{figure}

Figure ~\ref{fig:model:nh-amazon-results} shows the performance of a non-hierarchical network as tested from the Amazon test node.  The content response latency is characteristic of moving farther from the source node through the network.  The request nodes switch from Amazon infrastructure to Rackspace infrastructure starting with Neighbor (3), and this is reflected in the sudden increase in latency.  As the tests originate from Amazon, at the Neighbor (3) node, a request and it's response must travel from Virginia to Texas, then back to Virginia, then back to Texas, then back to the original requester in Virginia.  The spike in latency at Neighbor (3) when re-routing traffic is caused by SMTP delays with systems hosted at Google.  Overall, the distribution is very similar to the hierarchical case.  Also evident is a continuation of the previous pattern in which re-routing is the least efficient strategy, followed by encryption, then redaction.

Variance is generally low overall, with spikes associated with rerouting in the Neighbor (3) and (4) cases.  These spikes are associated with small groups of slow responses, where those responses are taking up to an order of magnitude more time to return than other, more typical samples.  This performance variability certainly effects the mean response times negatively, but even in samples with low variance, rerouting does perform the worst of the measured strategies.

%\begin{figure}[!t]
%\centering
%\includegraphics[width=6in]{nh_strategy_effects_rs}
%\caption{Non-Hierarchical  Results from Rackspace}
%\label{fig:model:nh-rackspace-results}
%\end{figure}

\begin{figure}[htbp]
\begin{minipage}[b]{0.5\linewidth}
\centering
\subfigure[Mean Response Times]{\includegraphics[width=\linewidth]{nh_strategy_effects_rs}}
\end{minipage}
\begin{minipage}[b]{0.5\linewidth}
\centering
\subfigure[Standard Deviation of Response Times]{\includegraphics[width=\linewidth]{nh_strategy_effects_stdev_rs}}
\end{minipage}
\caption{Non-Hierarchical Results from Rackspace}
\label{fig:model:nh-rackspace-results}
\end{figure}

Unlike the previous Amazon-based tests, the Rackspace tests shown in Figure ~\ref{fig:model:nh-rackspace-results} latencies seem much more uniform.  This again stems from the fact that each content query will always traverse the distance between Amazon and Rackspace data centers at least once.  Other than that, the distribution again shows an increase in measured latency as the queried node moves farther and farther away from the home node.  Once again, a dramatic spike in latency  associated with high performance variance emerges based on SMTP delays when rerouting information.  The pattern of rerouting having the highest latency continues here as well.  The dramatic variance in rerouting performance associated with the Neighbor (1) and (4) nodes results from small groups of samples with significantly degraded performance.  Even with this in mind, rerouting still performs the worst.

%\begin{figure}[!t]:q!

%\centering
%\includegraphics[width=6in]{nh_strategy_effects_local}
%\caption{Non-Hierarchical Results from Comcast}
%\label{fig:model:nh-comcast-results}
%\end{figure}

\begin{figure}[htbp]
\begin{minipage}[b]{0.5\linewidth}
\centering
\subfigure[Mean Response Times]{\includegraphics[width=\linewidth]{nh_strategy_effects_local}}
\end{minipage}
\begin{minipage}[b]{0.5\linewidth}
\centering
\subfigure[Standard Deviation of Response Times]{\includegraphics[width=\linewidth]{nh_strategy_effects_stdev_local}}
\end{minipage}
\caption{Non-Hierarchical Results from Comcast}
\label{fig:model:nh-comcast-results}
\end{figure}

Results from Comcast, included in Figure ~\ref{fig:model:nh-comcast-results}, shows a fairly regular distribution of response latencies overall.  In this case, the test node is ensconced within Comcast's network infrastructure.  Generally, re-routing is the least efficient approach, but not uniformly.  In this case, network effects created by the physical location of the testing node dominate these results.  We again have latency variance associated with rerouting resulting from a single sample with significantly degraded performance.

Non-hierarchical networks behave very similarly to hierarchical networks.  This is not surprising --- although the nodes are more functionally complex, performing routing and repository functions, once the content is found and delivered the roles the nodes fall into mirror those in a hierarchical network.  For example, in a typical query, a node will receive a request, check the repository for the requested content, and if the content does not exist, pass the request onto the next known nodes.  This does differ slightly from the hierarchical case in that the nodes check for content at each routing step, though this is a very fast and simple test.  Once content is found, the response is routed back the the requester without any repository checks, just as it would be in a hierarchical system.

\section{Removing Network Effects}
Having established the parameters under which confidentiality strategies may be chosen, the next immediate area of concern involves the number of filtering events that can occur prior to a given information network suffering from degraded performance.  Previous results demonstrated that some kind of degradation of performance in the selected network based on distance from content does exist, but that can also be attributed to the distributed nature of the network itself.  Processing performance of a given node must be evaluated free of network effects in order to more clearly understand the availability implications of content filtering itself.

A single node, configured on one of the test nodes in either infrastructure, would yield the type of network effect free performance limits needed.  A node in Amazon's environments was configured such that requests were made of the home node itself, under each of the three confidentiality strategies.  Requests were also directed to the home node without any usage management systems engaged in order to collect control data.  After that initial request, the node was configured with various usage management strategies in order to measure their availability impact.

%\begin{figure}[!t]
%\centering
%\includegraphics[width=6in]{single-node-results}
%\caption{Results from Requests to a Singe Node}
%\label{fig:model:single-node-results}
%\end{figure}

\begin{figure}[htbp]
\begin{minipage}[b]{0.5\linewidth}
\centering
\subfigure[Mean Response Times]{\includegraphics[width=\linewidth]{single-node-results}}
\end{minipage}
\begin{minipage}[b]{0.5\linewidth}
\centering
\subfigure[Assorted Other Statistics]{\includegraphics[width=\linewidth]{single-node-results-stats}}
\end{minipage}
\caption{Non-Hierarchical Results from Rackspace}
\label{fig:model:single-node-results}
\end{figure}

As shown in Figure ~\ref{fig:model:single-node-results}, with information network effects removed, redaction and encryption have very similar performance overall.  Redaction, as a strategy, is very simple programmatically, and as symmetric encryption is used for information protection, ciphering and deciphering operations are very fast.  Rerouting, in this case, is clearly the worst strategy.  This is a result of the dependency of this strategy on external systems and system configuration times.  Specifically, configuring and using SMTP for each rerouting operation is prohibitively expensive.  Rerouting as a strategy has intrinsic external dependencies, unlike other strategies measured.  This results in significantly variable performance of sample response times.  Single samples and small groups of samples exhibit significantly degraded performance in the collected experimental data, creating high standard deviation.  These significant outlier effects happen frequently enough to be a feature of the strategy rather than isolated events.
\section{Conclusions}
The work described herein presents bounds under which to select specific confidentiality strategies for protecting information in content networks.  We first described the state of the art of this kind of information protection in content networks, and introduced the current accepted protection architectures sponsored by the UCDMO.  We then presented a related taxonomy of increasing information protection, describing their advantages and disadvantages and how they could be implemented.  Next, we described our current customizable experimental framework for evaluating various confidentiality strategies.  We closed with a description of and the motivation for our experiments over these networks, the results of these experiments, and analysis of those results.  All simulation code is freely available via Github.

Overall, confidentiality strategy had little impact on information availability.  Redaction, rerouting, and encryption all performed within similar bounds.  Of these three approaches, redaction damaged information integrity the most, followed by rerouting, and then encryption, depending on the security of rerouting infrastructure.  Redaction provided the most confidentiality, followed by rerouting, and then by encryption (as encrypted content is generally at best a delaying tactic given enough time for cryptanalysis).  Based on these results, rerouting is likely the best general solution, depending on the existence of a secondary secure channel.  Less sensitive information can still be delivered via encryption, especially if that information is only sensitive within a given time window.  Very sensitive information can be redacted, but due to the related damage to integrity, this is only an attractive option when confidentiality is of the utmost importance.
	
Non-hierarchical and hierarchical networks performed similarly.  There was no significant difference in availability between networks with respect to confidentiality strategies.  Different network topologies certainly have different characteristics with respect to reliability as a result of selected architectures however, specifically with respect to the centralization or decentralization of key functions, but that analysis is outside the scope of this work.
	
\begin{table*}[tp] %
\centering %
\begin{tabular}{lccc}
\toprule %
{\it Property}			& {\it Redaction}	& {\it Rerouting} 	& {\it Encryption} 	\\\toprule
{\it Confidentiality} 	& 3				  	& 1					& 1				 	\\\midrule
{\it Integrity}			& 1					& 2					& 3 					\\\midrule
{\it Availability}		& 1					& 1					& 1					\\\bottomrule
\end{tabular}
\caption{Approach Evaluation Summary}
\label{table:model:evaluation}
\end{table*}

Table ~\ref{table:model:evaluation} shows the overall results of experiments and analysis with respect to various possible approaches to securing information transiting content networks, on a scale of one to three, with three the highest and one the lowest scores.  Not surprisingly, there is no clear best approach.  Rather, decisions with respect to which approach to choose for given content is highly dependent on the sensitivity of the content as well as integrity and availability requirements.
	
At this point, our information network implementation has integrated three different configurable strategies for information protection, and routes information via an overlay network using HTTP.  Longer term, this project will expand to both incorporate public-key encryption protocols and software defined networking (SDN) capabilities to provide physical control of information routing.  We intend to provide public-key encryption capabilities via an integrated public key infrastructure providing additional privacy and non-repudiation abilities for the network and SDN capabilities via integration with OpenFlow.  Shorter term goals include inclusion of different modes of operation, so that the network can support both request/response and publish/subscribe modes of operation, and more robust development so the system can run as a commercial grade security-on-demand service.


\pagebreak

\bibliographystyle{plain}
\bibliography{bib/proposal,bib/drm,bib/info-centric,bib/rfc}

\end{document}