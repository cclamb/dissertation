% \documentclass[12pt,letterpaper]{article}
\documentclass[botnum,fleqn,final]{unmeethesis}

% Set Title and Author
\newcommand{\mytitle}{Overlay Networks for Usage Management}
\newcommand{\myauthor}{Christopher C. Lamb}


\usepackage[latin1]{inputenc}
\usepackage{amsmath}
\usepackage{amsfonts}
\usepackage{amssymb}
%\usepackage{setspace}
\usepackage{graphicx}
\usepackage{epsfig}
\usepackage{booktabs}

% Font settings:
\usepackage[T1]{fontenc}
%\usepackage{mathpazo}
\usepackage{mathptmx}
\usepackage[scaled]{helvet}
\usepackage{courier}
\normalfont

% User microtype package to get some nicer font details
% http://www.ctan.org/tex-archive/macros/latex/contrib/microtype
\usepackage{microtype}

% Other packages
\usepackage{graphicx} % ability to include graphics
\usepackage[pdftex]{hyperref} % hyperrefs inside of PDF
\hypersetup{colorlinks,
            citecolor=black,
            filecolor=black,
            linkcolor=black,
            urlcolor=black,
            pdftitle={\mytitle},
            pdfauthor={\myauthor}}

% Use hyperref package to allow hyperlinks inside of the PDF document and PDF metadata
% http://en.wikibooks.org/wiki/LaTeX/Hyperlinks
\usepackage[pdftex]{hyperref}
\hypersetup{
    %bookmarks=true,          % show bookmarks bar?
    unicode=false,           % non-Latin characters in Acrobat’s bookmarks
    pdftoolbar=true,         % show Acrobat’s toolbar?
    pdfmenubar=true,         % show Acrobat’s menu?
    pdffitwindow=false,      % window fit to page when opened
    pdfstartview={FitH},     % fits the width of the page to the window
    pdftitle={\mytitle},     % title
    pdfauthor={\myauthor},   % author
    %pdfsubject={Subject},   % subject of the document
    %pdfcreator={Creator},   % creator of the document
    %pdfproducer={Producer}, % producer of the document
    %pdfkeywords={keyword1} {key2} {key3}, % list of keywords
    %pdfnewwindow=true,      % links in new window
    colorlinks=true,         % false: boxed links; true: colored links
    linkcolor=black,         % color of internal links
    citecolor=black,         % color of links to bibliography
    filecolor=black,         % color of file links
    urlcolor=blue            % color of external links
}


\graphicspath{{./images/}}

\begin{document}

\frontmatter

\title{\mytitle}
\author{\myauthor}

\degreesubject{Ph.D., Computer Engineering}

\degree{Doctor of Philosophy \\ Computer Engineering}

\documenttype{Dissertation}

\previousdegrees{B.S., Mechanical Engineering, New Mexico State University, 1994 \\
                 M.S., Computer Science, University of New Mexico, 2002}

\date{\today}

\maketitle

\makecopyright

\begin{dedication}
   Dedication.
\end{dedication}

\begin{acknowledgments}
   \vspace{1.1in}
   I would like to thank my advisor, Professor Gregory Heileman, for his support.
\end{acknowledgments}

\maketitleabstract

\begin{abstract}
Overlay networks have become a widely used technology with examples ranging from consumer focused distribution systems like BitTorrent to commercial content distribution systems like Akamai.  These kinds of systems, with the appropriate policy-centric content management components, can address looming problems in information distribution that both companies and federal agencies are beginning to face with respect to sensitive content.  This work addresses the current state of the art in these kinds of cross-domain systems, reviewing current example system architectures from the Unified Cross Domain Management Office (UCDMO), a federal organization specifically tasked with addressing these issues.  It then covers other related work, introduces a taxonomy of types of policy-centric usage managed overlay network systems and an associated methodology for evaluating the individual taxonomic elements.  It then delves into experimental evaluation of the various defined architectural options and finally presents results of comparing experimental evaluation with anticipated results. 
\end{abstract}

\tableofcontents
\listoffigures
\listoftables

\begin{glossary}{UHR}
   \item[RDFa] Resource Description Framework -- in -- attributes
   \item[XDM] Extensible Metadata Platform
   \item[XML] eXtensible Markup Language
\end{glossary}

\mainmatter

\chapter{Introduction}
\section{Introduction}
Current enterprise computing systems are facing a troubling future.  As things stand today, they are too expensive, unreliable, and information dissemination procedures are just too slow.

Generally, such systems still do not use current commercial resources as well as they could and use costly data partitioning schemes.  Most of these kinds of systems use some combination of systems managed in house by the enterprise itself rather than exploiting lower cost cloud-enabled services.  Furthermore, many of these systems have large maintenance loads imposed on them as a result of internal infrastructural requirements like data and database management or systems administration.  In many cases networks containing sensitive data are separated from other internal networks to enhance data security at the expense of productivity, leading to decreased working efficiencies and increased costs.

These kinds of large distributed systems suffer from a lack of stability and reliability as a direct result of their inflated provisioning and support costs.  Simply put, the large cost and effort burden of these systems precludes the ability to implement the appropriate redundancy and fault tolerance in any but the absolutely most critical systems.  Justifying the costs associated with standard reliability practices like diverse entry or geographically separated hot spares is more and more difficult to do unless forced by draconian legal policy or similarly dire business conditions.

Finally, the length of time between when a sensitive document or other type of data artifact is requested and when it can be delivered to a requester with acceptable need to view that artifact is prohibitively long.  These kinds of sensitive artifacts, usually maintained on partitioned networks or systems, require large amounts of review by specially trained reviewers prior to release to data requesters.  In cases where acquisition of this data is under hard time constraints like sudden market shifts or other unexpected conditional changes this long review time can result in consequences ranging from financial losses to loss of life.

Federal computer systems are prime examples of these kinds of problematic distributed systems, and demonstrate the difficulty inherent in implementing new technical solutions.  They, like other similar systems, need to be re-imagined to take advantage of radical market shifts in computational provisioning.

\section{Motivation}
Current policy-centric systems are being forced to move to cloud environments and incorporate much more open systems.  Some of these environments will be private or hybrid cloud systems, where private clouds are infrastructure that is completely run and operated by a single organization for use and provisioning, while hybrid clouds are combinations of private and public cloud systems.  Driven by both cost savings and efficiency requirements, this migration will result in a loss of control of computing resources by involved organizations as they attempt to exploit economies of scale and utility computing.

Robust usage management will become an even more important issue in these environments.  Federal organizations poised to benefit from this migration include agencies like the National Security Agency (NSA) and the Department of Defense (DoD), both of whom have large installed bases of compartmentalized and classified data.  The DoD realizes the scope of this effort, understanding that such technical change must incorporate effectively sharing needed data with other federal agencies, foreign governments, and international organizations \cite{proposal:info-sharing-strategy}.  Likewise, the NSA is focused on exploiting cloud-centric systems to facilitate information dissemination and sharing \cite{proposal:nsa-cloud}.

Cloud systems certainly exhibit economic incentives for use, providing cost savings and flexibility, but they also have distinct disadvantages as well.  Specifically, the are not intrinsically as private as some current systems, generally can be less secure than department-level solutions, and have the kinds of trust issues that the best of therapists cannot adequately address \cite{proposal:privacy-security-trust-cloud}.

As Pearson and Benameur \cite{proposal:privacy-security-trust-cloud} show, cloud technology is not currently as private as some organizations would like:
\begin{itemize}
\item \textit{User Data Control} --- In virtually any given Software-as-a-Service (SaaS) scenario, user data controls are sadly lacking.  Once data has been committed to a specific provider, that data is completely out of the original data owners control.  Furthermore, as we will see below, that data my not even be solely owned by the original owner anymore either.
\item \textit{Secondary Use} --- Most consumer facing social systems extensively mine user provided data for additional business advantages.  This is a common and well known secondary use for supplied data.  SaaS providers again have strong incentives to examine user provided information.
\item \textit{Offshore Development} --- Service users have no real control over who actually develops the systems a given service deploys.  Organizations have attempted to contractually limit development and support functions companies pursue to, say, the continental United States but have had very poor results with these kinds of unsupportable arrangements.
\item \textit{Data Routing} --- Both system providers and system users in fact have little control over routing issues.  Prohibiting data routing through sensitive countries is a difficult task for a single organization.
\item \textit{Secondary Storage} --- Most large-scale systems expect to use Content Delivery Networks (CDNs) to help manage content, and that expectation is heavily reflected in their physical system architectures. They simply cannot divorce use of CDNs from their systems for a single organization.
\item \textit{Bankruptcy and Data Ownership} --- Ownership and obligation to maintain expected data arrangements for a given company is not established under bankruptcy \cite{proposal:borders-info-I,proposal:borders-info-II,proposal:borders-info-III}.
\end{itemize}

Security issues also emerge from utility computing infrastructures:
\begin{itemize}
\item \textit{Data Access} --- System users have very little control over who, in the system provider's organization, is able to access their data and systems.
\item \textit{Data Deletion} --- Most savvy organizations have procedures in place to sanitize old storage elements like disk drives or backup tapes.  System users have very little control over if and how this is done when computing services are treated as a utility.
\item \textit{Backup Data Storage} --- Backup media is very difficult to encrypt, and most system providers still use tape systems as preferred media solutions for backup and storage needs.  These tapes, or copies of them, are generally stored offsite to support disaster recovery scenarios.  Security of these types of systems has been spotty to date \cite{proposal:saic-breach-I,proposal:saic-breach-II,proposal:saic-breach-III}.
\item \textit{Intercloud Standardization} --- Cloud computing systems do not have any standardized way to transfer computational units or data between systems.  Any protocols used for this kind of thing must be developed by customers themselves.  Due to the desire of providers to lock-in customers, this will likely not change as any standard development is strongly counter-incentiveized. 
\item \textit{Multi-tenancy and Side-Channels} --- Multi-tenant architectures in which multiple customers simultaneously use the same systems open those customers to covert side-channel attacks.
\item \textit{Logging and Auditing} --- Logging and auditing structures, especially for inter-cloud systems, are non-existent.
\end{itemize}

Finally, such systems suffer from internal and external trust issues:
\begin{itemize}
\item \textit{Trust Relationships} --- Trust is difficult to establish between individual cloud providers long-term.
\item \textit{Consumer Trust} --- Service users are still not entirely trusting of cloud system providers.
\end{itemize}

How to address these issues is an open research question.  Organizations ranging from cloud service providers to the military are exploring how to engineer solutions to these problems, and to more clearly understand the trade-offs required between selected system architectures \cite{proposal:assured-info-sharing}.  The problems themselves are wide ranging, appearing in a variety of different systems.  Military and other government systems are clearly impacted by these kinds of trust and security issues, and they also have clear information sensitivity problems.  This, coupled with the fact that these organizations have been dealing with these issues in one form or another for decades make them very well suited for prototypical implementation and study.

Current federal standards in place to deal with these issues in this environment are managed by the Unified Cross Domain Management Office (UCDMO).  UCDMO stakeholders range from the DoD to the NSA.  The current standard architectural model in place and governed by the UCDMO to deal with this kinds of issues are \textit{guard-centric cross domain architectures}.


%\chapter{Cross Domain Examples}
\section{Current Solutions}
\label{section:current-solutions}
Current federal standards in place to deal with these issues in this environment are managed by the Unified Cross Domain Management Office (UCDMO).  UCDMO stakeholders range from the DoD to the NSA.  The current standard architectural model in place and governed by the UCDMO to deal with these kinds of issues are guard-centric cross domain architectures.  As we show in section ~\ref{section:current-solutions}, the thinking behind these system architectures has remained relatively static over the past 20 years.  New thinking with regard to future internet architectures and usage management provide more powerful approaches to securing information as it flows through dynamic systems.

Current and near-future proposed solutions endorsed by the UCDMO include system architectures assembled by the NSA, Raytheon, and Booz $\mid$  Allen $\mid$  Hamilton (BAH).   The NSA has been active in this area for decades as a logical extension of their role in signals intelligence collection and processing.  Raytheon and BAH have been engaged over the past few years to provide an alternative voice and design approach to these kinds of systems, an effort met with limited success.

These cross-domain solutions are intended to enable sensitive information to easily flow both from a higher sensitivity domain to a lower sensitivity domain, and from lower to higher as well.  They generally act over both primary data (say, a document) and metadata over that primary data as well.  Note that in these system, in most cases, human intervention is still required to adequately review data prior to passing into lower security domains.

\subsection{NSA, Filtered}
The NSA conducted initial work in this area.  Their standard-setting efforts culminated in a reasonable conceptual system architecture, using groups of filters dedicated to specific delineated tasks to process sensitive information ~\cite{proposal:nsa-arch}.

\begin{figure}[!t]
\centering
\includegraphics[width=5in]{nsa-legacy-arch}
\caption{NSA Legacy Notional Architecture Model}
\label{fig:model:conceptual-model}
\end{figure}

In the scenario portrayed in Figure ~\ref{fig:model:conceptual-model}, \textit{Domain A} could very well be a private cloud managed by the U.S. Air Force, while \textit{Domain B} is a public operational network of some kind shared by coalition partners in a joint operation.

A system user attempts to send a \textit{data package} consisting of a primary document and associated metadata from \textit{Domain A} to \textit{Domain B}.  At some point, that submission reaches a \textit{guard}, which contains at least one \textit{filter chain}.  Each filter chain then contains at least one \textit{filter}.  Individual filters can execute arbitrary actions over a submitted data package and have access to any number of external resources as required.  At any point, a filter can examine the data package and reject it, at which point it will frequently wait for human review.  If a filter does not reject a data package, it passes that package onto the next filter or submits it for delivery to Domain B.

\subsection{NSA, Services}
In recent years, the NSA has extended the legacy system architecture for cross-domain information sharing to exploit service-oriented computing styles ~\cite{proposal:nsa-arch}.  Visualized in Figure ~\ref{fig:model:conceptual-model-services}, this model incorporates more modern conceptual elements and componentry.

\begin{figure}[!t]
\centering
\includegraphics[width=5in]{nsa-arch}
\caption{NSA Service-Oriented Model}
\label{fig:model:conceptual-model-services}
\end{figure}

In the view in Figure ~\ref{fig:model:conceptual-model-services}, we see on the left the \textit{Global Information Grid}, or \textit{GIG}.  On the right, we have the \textit{Distributed Service-oriented Cross Domain Solution}, or \textit{DSCDS}.  The GIG is not a truly open system --- rather, it is a loosely coupled collection of computational services handing data at a variety of levels of sensitivity, federated to provide stakeholders timely access to relevant information ~\cite{proposal:gig-arch}.  The DSCDS is essentially the embodiment of the NSA's cross-domain vision applied to service oriented computing.  This model fuses various technology choices with previous cross-domain thinking.

Indicative of this more modern system design thinking, we have a variety of services and service consumers attached to a common service bus within the GIG.  Within the DSCDS, we have groups of filters implemented as services inspecting transferred data when moved over the bus.  Finally, all of this interaction is managed by a management interface and controlled by an orchestration engine accessing a centralized group of policies.

Note that here we have begun to access a common policy repository for various types of security metadata regarding primary data elements.

\subsection{Raytheon}
In the past few years, Raytheon has offered a new model for cross domain use influenced by the NSA service-oriented model ~\cite{proposal:raytheon-arch}.

\begin{figure}[!t]
\centering
\includegraphics[width=5in]{raytheon-arch}
\caption{Ratheon Model}
\label{fig:model:conceptual-model-ray}
\end{figure}

The model in Figure ~\ref{fig:model:conceptual-model-ray}   is more grounded in the actual technical environment this kind of solution would be embedded within.  Here, we have the Non-secure Internet Protocol Router Network (NIPRNet) as one domain, and the Secret Internet Protocol Router Network (SIPRNet) as the other.  Here, NIPRNet is the lower security domain (lowside), and SIPRNet the higher security domain (highside).  This particular view shows the motion of data from the high side (SIPRNet) to the low side (NIPRNet).

Here, a data request is submitted from SIPRNet first two the \textit{XML Security Gateway} which calls into the \textit{Orchestration Engine} for policy validation.  The Orchstration Engine then coordinates calls into a \textit{Policy Repository} as well as to a collection of external \textit{Support Services}.  Once rectified against these elements, the request is passed into the \textit{Cross Domain Guard} which routes the request into the \textit{Unclassified Enclave} in NIPRNet.  Here, the request is passed directly through the lowside \textit{XML Security Gateway}, without rectification, onto the \textit{Service Provider}.  The response from the Service Provider is then passed back to the requester via the inverse path.

This model also begins to use a centralized policy repository, just as the NSA Service Model.  It also uses a single cross domain guard to transfer information from both the highside to the lowside, and vice-versa.

\subsection{Booz $\mid$ Allen $\mid$ Hamilton}
BAH submitted a competing model, also in 2009 ~\cite{proposal:bah-arch}.  In fact, both Raytheon and BAH presented their models under competitive contract to the UCDMO at the same conference, so the domain application is not coincidental.

\begin{figure}[!t]
\centering
\includegraphics[width=5in]{bah-arch}
\caption{Booz $\mid$ Allen $\mid$ Hamilton Model}
\label{fig:model:conceptual-model-bah}
\end{figure}

Figure ~\ref{fig:model:conceptual-model-bah} embodies BAH's thinking with respect to cross domain information management.  We have a \textit{Domain A} as a high security domain, and \textit{Domain B} as a low security domain.  Here, we again have dataflow from the highside to the lowside through the cross domain management system.

While not as detailed as the Raytheon proposal, this does have similar elements.  Here, we data first travels from Domain A into the \textit{Interface Segment for Domain A}, similar to the secret enclave used in the Raytheon model.  From there, it moves into the \textit{CI Segment}, which in turn submits the transferring data into the \textit{Filter Segment}.  From there, the package is moved into the \textit{Interface Segment for Domain B}, and then onto \textit{Domain B}.  The \textit{Administrative Segment} provides managment and oversight of the system as a whole.

Note the absence of specific policy-centric elements.  This system is reliant on specific policy-agnostic content filters as well.

%\chapter{Current Systems}
These kinds of cross-domain solutions still have clear similarities, and in fact have not progressed far beyond the initial notions of how these kinds of systems should work.  They still, for example, all use some kind of filter chaining mechanism to evaluate whether a given data item can be moved from a classified to an unclassified network.  Both NSA models used filters explicitly, as did the BAH model.  They all use a single guard as well, a sole point of security and enforcement, providing perimeter data security, but nothing else.  In each of these current system architectures, users are only allowed to exchange one type of information per domain.  The physical instantiations of these models are locked by operational policy to a single classification level.  Users cannot, for example, have Top Secret material on a network accredited for Secret material.  Finally, these models violate end-to-end principles in large service network design, centralizing intelligence rather than pushing that intelligence down to the ends of the system ~\cite{Clark:1995:DPD:205447.205458}.

End-to-end principles are generally considered core to the development of extreme scale, distributed systems.  Essentially, one of the key design decisions with respect to the early internet was to move any significant processing to system end nodes, keeping the core of the network fast and simple.  Known as the end-to-end principles, this design has served the internet well, allowing it to scale to sizes unconceived when originally built.  Current cross domain systems are placed at key routing points between sensitive networks.  These locations are core to information transfer between systems and as a result violate the initial design principles upon which the internet was founded.  There does exist some belief that end-to-end principles need to be modified to support future networks, but nevertheless, current cross domain systems still violate the basic ideas behind large, scalable networks by placing complex application-specific logic directly and only in the core of a given sensitive network ~\cite{Blumenthal:2001:RDI:383034.383037}.

Future systems will generally demonstrate decentralized policy management capabilities, infrastructural reuse, the ability to integrate with cloud systems, and security in depth.  Policy management will need to be decentralized and integrated within the fabric of the system.  The system is both more secure and resilient as a result, better able to control information and operate under stressful conditions.  Multi-tenancy can lower costs and increase reliability and is furthermore a common attribute of cloud systems.  An appropriately secured system facilitates integration of computing resources into multi-tenant environments.  The ability to handle multi-tenant environments and to reliably secure both data at rest and data in motion leads to computational environments deployable in cloud systems.  Finally, systems must operate under \textit{all} conditions, including when they are under attack or compromise ~\cite{proposal:ron-ross} and provide protection to sensitive data in depth.


\chapter{Proposed Taxonomy}
\section{Taxonomies of Usage Management Overlay}
A clear taxonomic organization of potential steps in approaching finer grained policy based usage management helps in describing the difficulties inherent in developing potential solutions as well as aiding in planning system evolution over time. Here, we have five distinct types of integrated policy-centric usage management systems, as shown in Table \ref{table:model:taxonomy}.  Of these five, only the first two levels are represented in current system model.

\begin{table}[tp] %
\centering %
\begin{tabular}{clcc}
\toprule %
$ Name$ 	& $Description$ \\\toprule %
$\phi$ 		& The initial level of this taxonomy, $\phi$ classified systems \\
 			& have a single guard without policy-based control \\\midrule
$\alpha$	& $\alpha$ classified systems have a single guard by have begun \\
			& to integrate policy-based control \\\midrule
$\beta$		& Systems that have begun to integrate policy-based control with \\
			& router elements are in the $\beta$ category \\\midrule
$\gamma$	& Systems that have integrated policy-based control with routing \\
			& and computational elements \\\midrule
$\delta$	& Continuous policy-based control with \textit{smart licensed} artifacts \\\bottomrule
\end{tabular}
\caption{Proposed Usage Management Taxonomy}
\label{table:model:taxonomy}
\end{table}

In this taxonomy, it is not required that systems pass through lower levels to reach higher ones.  This taxonomy represents a continuum of integration of usage management controls.  Systems can very well be designed to fit into higher taxonomic categories without addressing lower categories.  That said however, many of the supporting infrastructural services, like identification management or logging and tracing systems, are common between multiple levels.

The taxonomy itself starts with the current state, integrating policy evaluation systems into the network fabric gradually, moving away from filters, then by adding policy evaluation into the routing fabric, then the computational nodes, and finally by incorporating evaluation directly into content.

\subsection{$\phi$-level Overlay Systems}

\subsection{$\alpha$-level Overlay Systems}

\subsection{$\beta$-level Overlay Systems}

\subsection{$\gamma$-level Overlay Systems}

\chapter{Metric Selection}
\subsection{Evaluation Methodology and Model}
In order to successfully evaluate the elements of our overlay taxonomy, we must first establish a model against which to measure the presented architectures.  The current standard for evaluating software quality is ISO/IEC 25020 and this, along with other related standards from other service delivery organizations has begun to be integrated into both academia and industry as a tractable way to measure system quality \cite{5958158, proposal:iso-25020}.

This particular model must address quality attributes specific to the presented architectures rather than the functional domain.  The goal of this model is to allow for architectural evaluation of policy evaluating architectures regardless of the specific functional domain.  Ergo, injecting a specific functional domain into the evaluation or the evaluating model is unacceptable.  Acceptable attributes are those which directly target quality attributes of the architectures in question.

\begin{eqnarray}
E = \lbrace f_{e}, f_{r}, f_{u}, f_{p}, f_{m}, f_{f}, f_{s}, f_{c} \rbrace \\
W= \lbrace w_{e}, w_{r}, w_{u}, w_{p}, w_{m}, w_{f}, w_{s}, w_{c} \rbrace \\
s = \sum_{W, E} w_{i} f_{i}
\end{eqnarray}

We are specifically interested in evaluating architectures for policy evaluation functional suitability, reliability, usability, possible performance efficiencies, maintainability, portability, security, and compatibility, specifically neglecting any kind of domain functional suitability.  Each area will be associated with an evaluation function.  The suitability of a given architectural option will be evaluated by a tuple of these functions, which can then be converted into a weighted sum leading to a single quantitative metric representing suitability under evaluated conditions.

Also important to note, certain attributes may not be able to be evaluated using specific architectural models.  For example, conceptual or notional architecture models are intended to convey specific ideas prevalent in a given system architecture rather than ways that architecture will be realized.  As such, these kinds of models generally cannot be evaluated for things like portability or performance efficiency, as these qualities usually manifest based on specific standard and technology selections, respectively.  In these cases, the evaluation functions representing those attributes will be weighted at zero.

\subsubsection*{Functional Suitability}
Functional suitability in this context is reflects the ability of the system to accurately manage artifacts based on policy and context.  As the functional domain is the same for all examined systems, we will neglect this and set $ w_{e} = 0 $.

\subsubsection*{Reliability}
Here, we will evaluate reliability via a simple Bayesian Belief Networks and Reliability Theory.  The function $ f_{r} $ for a given architecture will be a functional representation of the independent variables required to evaluate the network.

\subsubsection*{Usability}
These are purely notional models, and as a result cannot be evaluated using generally accepted usability metrics like the Systems Usability Scale (SUS) \cite{proposal:sus}.  Ergo, for this analysis we will set the weight $w_{u} = 0$.

\subsubsection*{Performance Efficiency}
Performance efficiency is generally a characteristic of logical and physical system architectures.  Specific logical architectures can certainly decrease the performance of a given system through poor design, insufficient caching, badly considered state management, or inferior scalability.  Physical architectures clearly impact performance if processing power, storage, communication bandwidth, or other attributes are insufficient to process apparent loads.  We will therefore set $w_{p} = 0$ in this analysis.

\subsubsection*{Maintainability}
Maintainability can be measured via examination of a given system with an eye toward areas prone to change.  Loosely coupled components directly contribute to the ability to change a specific system component.  With this in mind, we can measure develop a simple ratio to indicate the maintainability of a given system --- the total number of components with high change impact to the number of components with high change impact that have been decoupled from the system via an interface 

\begin{equation}
f_{m} = c_{i} / c_{\delta}
\end{equation}

\subsubsection*{Portability}
One way to evaluate system portability is via standard compliance \cite{5958158}.  Proposed architectures herein will not be evaluated to the level at which proposed standards have an impact.  Therefore, in this analysis, $ w_{f} =0 $.

\subsubsection*{Security}
We will evaluate security using Reliability Theory and Bayesian Belief Networks as well.  In that network we will evaluate confidentiality, integrity, and availability failure modes.  We will define $ f_{s} $ as we did $ f_{r} $ previously.

\subsubsection*{Compatibility}
Compatibility with existing systems requires either accepted system standards or other systems with which to be compatible.  In this analysis, we have neither accepted standards or other systems, so we will set $ w_{c} = 0 $.

Any weights not explicitly set to zero previously will be set to one, giving us:

\begin{equation}
W = \lbrace 0, 1, 0, 0, 1, 0, 1, 0 \rbrace
\end{equation}

or, in essence:

\label{equation:final}
\begin{equation}
s = f_{r} + f_{m} + f_{s}
\end{equation}




As Pearson and Benameur \cite{proposal:privacy-security-trust-cloud} show, cloud technology is not currently as private as some organizations would like:
\begin{itemize}
\item \textit{User Data Control} --- In virtually any given Software-as-a-Service (SaaS) scenario, user data controls are sadly lacking.  Once data has been committed to a specific provider, that data is completely out of the original data owners control.  Furthermore, as we will see below, that data my not even be solely owned by the original owner anymore either.
\item \textit{Secondary Use} --- Most consumer facing social systems extensively mine user provided data for additional business advantages.  This is a common and well known secondary use for supplied data.  SaaS providers again have strong incentives to examine user provided information.
\item \textit{Offshore Development} --- Service users have no real control over who actually develops the systems a given service deploys.  Organizations have attempted to contractually limit development and support functions companies pursue to, say, the continental United States but have had very poor results with these kinds of unsupportable arrangements.
\item \textit{Data Routing} --- Both system providers and system users in fact have little control over routing issues.  Prohibiting data routing through sensitive countries is a difficult task for a single organization.
\item \textit{Secondary Storage} --- Most large-scale systems expect to use Content Delivery Networks (CDNs) to help manage content, and that expectation is heavily reflected in their physical system architectures. They simply cannot divorce use of CDNs from their systems for a single organization.
\item \textit{Bankruptcy and Data Ownership} --- Ownership and obligation to maintain expected data arrangements for a given company is not established under bankruptcy \cite{proposal:borders-info-I,proposal:borders-info-II,proposal:borders-info-III}.
\end{itemize}

Security issues also emerge from utility computing infrastructures:
\begin{itemize}
\item \textit{Data Access} --- System users have very little control over who, in the system provider's organization, is able to access their data and systems.
\item \textit{Data Deletion} --- Most savvy organizations have procedures in place to sanitize old storage elements like disk drives or backup tapes.  System users have very little control over if and how this is done when computing services are treated as a utility.
\item \textit{Backup Data Storage} --- Backup media is very difficult to encrypt, and most system providers still use tape systems as preferred media solutions for backup and storage needs.  These tapes, or copies of them, are generally stored offsite to support disaster recovery scenarios.  Security of these types of systems has been spotty to date \cite{proposal:saic-breach-I,proposal:saic-breach-II,proposal:saic-breach-III}.
\item \textit{Intercloud Standardization} --- Cloud computing systems do not have any standardized way to transfer computational units or data between systems.  Any protocols used for this kind of thing must be developed by customers themselves.  Due to the desire of providers to lock-in customers, this will likely not change as any standard development is strongly counter-incentiveized. 
\item \textit{Multi-tenancy and Side-Channels} --- Multi-tenant architectures in which multiple customers simultaneously use the same systems open those customers to covert side-channel attacks.
\item \textit{Logging and Auditing} --- Logging and auditing structures, especially for inter-cloud systems, are non-existent.
\end{itemize}

Finally, such systems suffer from internal and external trust issues:
\begin{itemize}
\item \textit{Trust Relationships} --- Trust is difficult to establish between individual cloud providers long-term.
\item \textit{Consumer Trust} --- Service users are still not entirely trusting of cloud system providers.
\end{itemize}

\chapter{Experimental Configuration}
\begin{figure*}[!t]
\centering
\includegraphics[width=6in]{cross-domain-prototype}
\caption{Simulation Logical Configuration}
\label{fig:model:cross-domain-prototype}
\end{figure*}

\section{Overlay Implementation Concerns}
A key concept in our current work is the separation of content management from physical communication networks.  In the past, content was controlled via partitioning and physical network access management.  Physical networks were tightly controlled as a way to manage access to sensitive content.  Classified networks in common use today are canonical examples of this kind of approach to content management.  Access to these networks is tightly controlled by classification authorities and the ability to transfer content from these networks to more open systems is rigorously managed.  Corporate systems have also commonly used this kind of approach, though not usually with so much regulation or rigor.

This kind of approach is not scalable however.  It imposes huge costs and infrastructural requirements that are becoming too large to effectively manage.  Furthermore, future systems containing sensitive information require similar security features, and simply cannot be developed without custom controlled infrastructure.  Health care systems, for example, have huge security needs and a more finely grained level of application than even deployed government systems.  These systems will contain exabytes of data, all of which needs to be explicitly controlled, managed, and reviewed by those associated with specific managed records.

Separating content networks from physical networks enables network infrastructure virtualization and multi-tenancy.  Use the popular file-sharing system BitTorrent as an example.  BitTorrent is a content network optimized for download efficiency.  It run over traditional TCP/IP networks, but manages traffic according to specialized algorithms unique to BitTorrent.  These algorithms take advantage of the asymmetry between upload and download speeds of typical home-use Internet systems in which upload speeds are regularly an order of magnitude slower than download speeds.  By partitioning content into distinct sections and downloading them from multiple clients, a downloading node can effectively use all available download bandwidth and is no longer necessarily constrained by the upload bandwidth of a serving peer system.  We use a similar approach, in that our hypothesized systems also overlay TCP/IP traffic, but rather than optimizing download speeds we focus on content usage management.

Just as systems like BitTorrent runs over current established protocols, usage management overlay systems could as well.  They support multi-tenant cloud computing systems by providing secure compartmentalized access to managed information.  They also support the ability to create and use integrated overlay systems between multiple cloud providers, supporting running of overlay components in systems hosted at Amazon while accessing nodes executing on Rackspace infrastructure.

Content networks must deal with situations analogous to those encountered in previous physical systems.  Specific examples include cross-domain monitoring and content mashing.  Both problems are currently areas of active research within physical networks and need extensive examination in overlay systems as well.

To begin with, in content-specific overlay networks, cross-domain routing can become an even more pervasive issue.  Currently, cross-domain data processing guards are installed on the perimeter of sensitive networks where they can monitor and manage outgoing and incoming traffic.  In content networks, these kinds of systems can begin to multiply within the information transmission fabric.  In physical networks, the network topology is fixed and is established when the network is installed.  After installation, changes in the essential network topology are cost-prohibitive and correspondingly rare.  Overlay systems do not suffer from this high cost of change, and can easily morph from one topology to another.  As additional content enclaves appear within a given overlay topology, the need for content usage management between those enclaves increases.

Mashup scenarios become similarly common.  As additional sources of accessible data appear, opportunities for inappropriate data combinations increase at best geometrically.  Data combinations need to be likewise managed to prevent inappropriate data combinations.

\section{Initial Prototype Implementation}
Our first completed prototype shows that overlay routers can in fact use licenses bundled alongside content to modify transmitted content based on dynamic network conditions.  Running on a single host over HTTP, it simulates two content domains and communication between them.  The communication link has uncertain security state and changes over time.  Note that this prototype currently runs on a single host with varying ports, but it could easily run on multiple hosts as well.  The current single host configuration is simply to simplify system startup and shutdown.

License bundles are hosted on the filesystem, though they could be hosted in any other data store.  These artifacts are currently XML.  They are stored in a directory, and the license file has a LIC extention while the content file has an XML extension.  Both the content and the license files have the name of the directory in which they reside (for example, if the directory is named test, the license file is named test.lic and the content file test.xml).  In this context, the directory is the content bundle.  The license and content files are simply documents and port to document-centric storage systems like MongoDB easily.  They can certainly be stored in traditional relational databases as well.

The system itself has two domains, Domain 0 and Domain 1.  Each domain consists of a client node and a content router node.  Requests are initially served to client nodes.  If client nodes do not contain the requested content, they the forward that request to their affiliated content router.  The content router will send that request to all the content routers of which it is aware.  Those other routers will then query associated client nodes for content.  If the requested content is in fact found, it will be returned to the original requesting router and then to the requesting node.  If the content is not found, HTTP status 404 codes are returned to requesting routers and nodes.

All router-to-router content traffic is modified based on security conditions.  A Context Manager maintains metadata regarding network paths.  If a given network path is only cleared for data of a certain sensitivity level, a transmitting router will remove all license information and content that is associated with higher sensitivities, and then transmit only information at an appropriate sensitivity level over the link.

Figure \ref{fig:model:cross-domain-prototype} shows the prototypical workflow through the system across the domains, and Figure \ref{fig:model:prototype-physical-config} shows the current system configuration of the simulation, with the cross-domain link highlighted in red.  The system is current configured to use ports 4567 through 4571.

All content requests are via HTTP GET.  Link status can be changed via HTTP POST and we use the CURL command to exercise the network.

This proof-of-concept does implement a simple overlay network for usage managed content over HTTP, easily extensible to HTTPS.  Changes in the context of the network dynamically change the format of transmitted content.  All source code for this simulation is publically available on GitHub, at https://github.com/cclamb/overlay-network, with documentation on how to run the simulation.

\begin{figure}[!t]
\centering
\includegraphics[width=3in]{prototype-physical-config}
\caption{Physical Simulation Configuration}
\label{fig:model:prototype-physical-config}
\end{figure}

\section{Initial Prototype Results}

Policy and content delivery over HTTP possible

Ruby/Sinatra will support HTTP overlay development, CURL for usage

Information Filtering Expectations

Extension into larger distributed system feasible

\section{Inter-Provider Cloud Configuration}
Over the month of June we established our initial technical baseline for upcoming content network development.  We have created and deployed baseline system images in both Amazon's Elastic Compute Cloud (EC2) and Rackspace Servers infrastructures.  We have also created and exercised our deployment, configuration, and logging systems to enable distributed monitoring and centralized reporting.  Overall, we currently have 20 nodes running with two distinct providers geographically dispersed across the continental United States.  This leads to a distinct requirement for a centralized system with distributed access for both initial configuration information as well as logging and auditing.  We have implemented this required infrastructure using Amazon's Simple Storage Service (S3), accessible from both Rackspace and Amazon hosted virtual machines.
The specific technical components are Amazon EC2, Amazon S2, Rackspace Servers, and GitHub.  Both EC2 and Rackspace nodes are Ubuntu virtual machines, albeit at different versions, as we run Ubuntu version 11.04 in Rackspace and Ubuntu Version 12.04 in Amazon's infrastructures.  These systems are provisioned with Git, Ruby, the Ruby Version Manager (RVM), and supporting libraries.  They all run as micro-intances or equivalent, and are bootstrapped with the appopriate project information to begin to participate as an overlay network node.  While EC2 and Rackspace Server infrastructures are infrastructure-as-a-cloud (IaaS) offerings supporting virtual machine instances of various types, Amazon S3 is a simple key-value store.  Running with REST sematics over HTTP, S3 stores arbitrary documents associated with specific keys in buckets.  These documents can be downloaded by any authorized participant, where authoriztion state is proven by possention of a secret key.  In this way, we can store the global configuration of a specific overlay network in a single location from which every node can access informationm with respect to their pending role and needed configuraiton information.  Likewise, all overlay network state can also be saved to centralized buckets for later analysis.  Finally, Github is a centralized source code repository used to share code between all participating nodes.  Prior to each content network instantiation, each node checks the repository for updates, and downloads them if they exist.

All data saved within S3 is serialized in a text-based data serialization language known as YAML.  YAML is a widely supported hierarchical data representation language with support within the Ruby core platform.  This enables us easily serialize Ruby-native data structures to text-based representations for storage within S3.  More importantly, it simplifies post-experimental data analysis as any information logged to the centralized logging system during a given experimental run can be easily read and analyzed after the fact.

In order to manage and initialize all overlay nodes, we use Capistrano.  Capistrano is a distributed deployment system initially used to manage large clusters of Ruby-on-Rails systems.  It has since expanded into a general-purpose distributed deployment toolchain, tightly integratd with Git.  This allows us to bootstrap different configurations of networks from a single command-and-control node simply and efficiently.

\begin{table*}[tp] %
\centering %
\begin{tabular}{clcc}
\toprule %
$Category$ 				& $Components$ 								\\\toprule %
$Infrastructure$ 		& Amazon S3, Amazon EC2, Rackspace Servers 	\\\midrule
$Operating Systems$		& Ubuntu 11.04, Ubuntu 12.04 				\\\midrule
$Technologies$			& Ruby (Sinatra, Capistrano, YAML) 			\\\midrule
$Supporting Systems$	& Git, Github 								\\\bottomrule
\end{tabular}
\caption{Supporting Components}
\label{table:model:components}
\end{table*}

All these infrastructural elements, protocols, and technology components have been successfully tested, allowing for unified control and configuration of large, distributed overlay systems.  We have successfully tested our logging systems, and integrated them with the Ruby runtime for ease of access.  We have also passed configuration information to both Rackspace and Amazon EC2 systems and verified access from all participating nodes.  Finally, we have successfully exercised the ability to dynamically update all participating nodes from Github as well as the capability to manage the system via Capistrano.



\pagebreak

\bibliographystyle{plain}
\bibliography{bib/proposal,bib/drm}

\end{document}