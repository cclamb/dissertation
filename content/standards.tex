\section{National and International Standards}
Once the Bell-LaPadula and Biba models were in developed and understood, the Department of  started another effort to more clearly categorize the security of information systems and related networks.  This effort cumulated in what is now known as the Rainbow Series of books, specifically the {\it Trusted Computer System Evaluation Criteria}, known as the Orange Book, and the {\it Trusted Network Interpretation}, or Red Book.

The Orange Book addresses security of individual computer systems.  It establishes a taxonomy of security levels at which systems can be classified.  These levels range from systems with minimal security characteristics and stretch to formally verified computer systems.  These levels establish such characteristics as separate administrator accounts, role-based security systems, data labeling, and the recognition of possible covert communications channels ~\cite{OrangeBook}.

The Red Book deals with how networks protect the confidentiality, integrity, and availability of information.  It addresses encryption, digital signatures, flow protection and confidentiality, data confidentiality, and infrastructure protection.  Like the Orange Book, it is a framework established over a set of principles to facilitate network security.  It discusses how subjects and objects need to be controlled and accounted for in computer networks just as they are within individual computer systems ~\cite{RedBook}.

Today, the Rainbow Series has largely been supplanted by the International Organization for Standardization's Common Criteria.  The Rainbow Series was seen as too rigid, and other standards too flexible and difficult to implement, leading to the development of the Common Criteria in 1993.  The Common Criteria are based on Rainbow Series, the Canadian Trusted Computer Product Evaluation Criteria (CTCPEC), and the Information Technology Security Evaluation Criteria (ITSEC).  While the Orange Book examined a system based on a Bell-LaPadula-centric perspective, the Common Criteria evaluates computing systems based on a pre-established protection profile, designed to cover a specific security requirement.  The common criteria also use a taxonomy to classify systems.  This taxonomy has seven levels, again ranging from minimally tested systems that have some assured functionality to formally verified, designed, and tested information systems ~\cite{CommonCriteria}.

Other standards commonly referenced include publications from the National Institute of Standards and Technology, particularly from their 800-series of special publications.  These publications cover everything from cloud computing security ~\cite{NIST800:144} to single computer systems \cite{NIST800:12} to web services \cite{NIST800:95}.