\label{section:capabilities}
\color{red}
Need more detail here expanding and establishing claims.  Examples on why layer breaking is bad.  Policy size being too large for single packets, and why that is bad.  Outline why re-assembling large documents for detailed analysis is hard, and needed when information spans packets.  Add cites.
\color{black}

Usage management is better served by information-centric networks than traditional packetized networks.  The basic structure of packet networks facilitates simple and efficient data transfer, but is fundamentally based on certain design assumptions that render network-centric usage management difficult at best and impossible at worst ~\cite{Cl:88,SaReCl:84}.  Information-centric networks, taking a very different approach to network design, are much more amenable to embedded content control based on their different design principles ~\ref{section:information-centric-networking}.

Current packet-based systems share two common design principles.  Strict layering, in which upper layers only use services that exist in lower layers which in turn have no knowledge of upper layers, and limited runtime packet sizes.  In internet systems, switching and routing occur in the lower layers of the OSI model (layers two and three) ~\cite{TODO}.  These decisions are made based on a priori knowledge of a given network topology, and are not impacted by transmitted content ~\cite{TODO}.  In fact, access to application content occurs at much higher levels (specifically, layer seven).  As a result of strict service layering, the content information needed to make content-sensitive routing decisions is simply not available without breaking layer encapsulation.  Likewise, policies associated with content can be arbitrarily large ~\cite{TODO}.  As a result, they can exceed maximum packet sizes defined in packetized networks.  Furthermore, as content-sensitive networks must evaluate defined policies prior to routing content, any policy to be evaluated must be completely downloaded into a router and analyzed for suitability for transmission prior to any packet routing, leading to inevitable bottlenecks as content is queued behind the policy elements.

Information-centric neworks are based on different primitives ~\ref{section:information-centric-networking}.  Specifically, they are based on named data objects with strict name-data integrity, as well as other associated principles.  This different abstraction makes policy evaluation and content binding simpler, as content can be bound either in-line to policy or via specific naming conventions.  In these systems, once content is located by name, it is returned to the requester either via a predefined path mirroring the original request path or a variable response path.  In either case however, all content and associated policy is available at each routing node in that return path, and can be evaluated for suitability of transmission.

As a result of these fundamentaly different underlying models, information-centric networks in the next-generation internet enable usage management capabilities that are very problematic to implement and enforce in current internet architectures without dedicated application-level overlays.