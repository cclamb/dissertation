\subsection{Cross Domain Solutions}
The Unified Cross Domain Management Office (UCDMO) supports efforts to develop other specific solutions that have been presented over the past few years to handle this kind of information management.  The National Security Agency set the standard in this area initially.  In 2009, at a conference sponsored by the UCDMO, Booz $\mid$ Allen $\mid$ Hamilton and Raytheon presented alternative notional architectures contrasting with current NSA-influenced approaches \cite{proposal:nsa-arch,proposal:gig-arch,proposal:bah-arch,proposal:raytheon-arch}.

These kinds of cross-domain solutions still have clear similarities, and in fact have not progressed far beyond the initial notions of how these kinds of systems should work.  They still, for example, all use some kind of filter chaining mechanism to evaluate whether a given data item can be moved from a classified to an unclassified network.  Both NSA models used filters explicitly, as did the BAH model.  They all use a single guard as well, a sole point of security and enforcement, providing perimeter data security, but nothing else.  In each of these current system architectures, users are only allowed to exchange one type of information per domain.  The physical instantiations of these models are locked by operational policy to a single classification level.  Users cannot, for example, have Top Secret material on a network accredited for Secret material.  Finally, these models violate end-to-end principles in large service network design, centralizing intelligence rather than pushing that intelligence down to the ends of the system \cite{Blumenthal:2001:RDI:383034.383037}.

Future systems will generally demonstrate decentralized policy management capabilities, infrastructural reuse, the ability to integrate with cloud systems, and security in depth.  Policy management will need to be decentralized and integrated within the fabric of the system.  The system is both more secure and resilient as a result, better able to control information and operate under stressful conditions.  Multi-tenancy can lower costs and increase reliability and is furthermore a common attribute of cloud systems.  An appropriately secured system facilitates integration of computing resources into multi-tenant environments.  The ability to handle multi-tenant environments and to reliably secure both data at rest and data in motion leads to computational environments deployable in cloud systems.  Finally, systems must operate under \textit{all} conditions, including when they are under attack or compromise \cite{proposal:ron-ross}.  Ergo, they must provide protection to sensitive data in depth.

\subsection{Other Related Work}
This work introduces the notion of usage management embedded in a delivery network itself.  It also provides an in-depth analysis of the challenges and principles involved in the design of an open, inter-operable usage management framework that operates over this kind of environment. Besides referencing the material we have covered in depth to portray the current state of the art, the analysis includes application of well-known principles of system design and standards~\cite{BlCl:01,Cl:88,ClWrSoBr:02}, research developments in the areas of usage control~\cite{PaSa:04,JaHeLa:10}, policy languages design principles~\cite{JaHeMa:06}, digital rights management (DRM) systems~\cite{JaHe:09},  and interoperability~\cite{JaHe:04,HeJa:05,KoLaMaMi:04,coral,marlin} towards the development of supporting frameworks.

While a large body of work exists on how overlay networks can use policies for \textit{network} management, very little work has been done on using usage policies for \textit{content} management.  The primary contribution in this area focuses on dividing a given system into specific \textit{security domains} which are governed by individual policies \cite{4457175}.  This system fits into our proposed taxonomy as an $\alpha$-type system as it has domains with single separating guards.

A large body of work currently exists with respect to security in and over overlay networks.  These kinds of techniques and this area of study is vital to the production development and delivery of overlay systems, but is outside the scope of this work.