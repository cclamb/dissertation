\section{Conclusions}
The work described herein presents bounds under which to select specific confidentiality strategies for protecting information in content networks.  The state of the art of this kind of information protection in content networks was first described, and the current accepted protection architectures sponsored by the UCDMO were introduced.  A related taxonomy of increasing information protection was then presented, describing their advantages and disadvantages and how they could be implemented.  Next, the current customizable experimental framework for evaluating various confidentiality strategies was described.  A description of and the motivation for the experiments over these networks, the results of these experiments, and analysis of those results closed the dissertation.  All simulation code is freely available via Github.

Overall, confidentiality strategy had little impact on information availability.  Redaction, rerouting, and encryption all performed within similar bounds.  Of these three approaches, redaction damaged information integrity the most, followed by rerouting, and then encryption, depending on the security of rerouting infrastructure.  Redaction provided the most confidentiality, followed by rerouting, and then by encryption (as encrypted content is generally at best a delaying tactic given enough time for cryptanalysis).  Based on these results, rerouting is likely the best general solution, depending on the existence of a secondary secure channel.  Less sensitive information can still be delivered via encryption, especially if that information is only sensitive within a given time window.  Very sensitive information can be redacted, but due to the related damage to integrity, this is only an attractive option when confidentiality is of the utmost importance.
	
Non-hierarchical and hierarchical networks performed similarly.  There was no significant difference in availability between networks with respect to confidentiality strategies.  Different network topologies certainly have different characteristics with respect to reliability as a result of selected architectures however, specifically with respect to the centralization or decentralization of key functions, but that analysis is outside the scope of this work.
	
\begin{table*}[tp] %
\centering %
\begin{tabular}{lccc}
\toprule %
{\it Property}			& {\it Redaction}	& {\it Rerouting} 	& {\it Encryption} 	\\\toprule
{\it Confidentiality} 	& 3				  	& 1					& 1				 	\\\midrule
{\it Integrity}			& 1					& 2					& 3 					\\\midrule
{\it Availability}		& 1					& 1					& 1					\\\bottomrule
\end{tabular}
\caption{Approach Evaluation Summary}
\label{table:model:evaluation}
\end{table*}

Table ~\ref{table:model:evaluation} shows the overall results of experiments and analysis with respect to various possible approaches to securing information transiting content networks, on a scale of one to three, with three the highest and one the lowest scores.  Not surprisingly, there is no clear best approach.  Rather, decisions with respect to which approach to choose for given content is highly dependent on the sensitivity of the content as well as integrity and availability requirements.
	
At this point, the information network implementation has integrated three different configurable strategies for information protection, and routes information via an overlay network using HTTP.  Longer term, this project will expand to both incorporate public-key encryption protocols and software defined networking (SDN) capabilities to provide physical control of information routing.  Public-key encryption capabilities via an integrated public key infrastructure providing additional privacy and non-repudiation abilities for the network and SDN capabilities via integration with OpenFlow will be investigated.  Shorter term goals include inclusion of different modes of operation, so that the network can support both request/response and publish/subscribe modes of operation, and more robust development so the system can run as a commercial grade security-on-demand service.
