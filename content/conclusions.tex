\section{Conclusions}
The work described in this paper presents bounds under which to select specific confidentiality strategies for protecting information in content networks.  We first described the state of the art of this kind of information protection in content networks, and introduced the current accepted protection architectures sponsored by the UCDMO.  We then presented a related taxonomy of increasing information protection, describing their advantages and disadvantages and how they could be implemented.  Next, we described our current customizable experimental framework for evaluating various confidentiality strategies.  We closed with a description of and the motivation for our experiments over these networks, the results of these experiments, and analysis of those results.  All simulation code is freely available via Github.

	Overall, confidentiality strategy had little impact on information availability.  Redaction, rerouting, and encryption all performed within similar bounds.  Of these three approaches, redaction damaged information integrity the most, followed by rerouting, and then encryption, depending on the security of rerouting infrastructure.  Redaction provided the most confidentiality, followed by rerouting, and then by encryption (as encrypted content is generally at best a delaying tactic given enough time for cryptanalysis).  Based on these results, rerouting is likely the best general solution, depending on the existence of a secondary secure channel.  Less sensitive information can still be delivered via encryption, especially if that information is only sensitive within a given time window.  Very sensitive information can be redacted, but due to the related damage to integrity, this is only an attractive option when confidentiality is of the utmost importance.
	
	At this point, our information network implementation has integrated three different configurable strategies for information protection, and routes information via an overlay network using HTTP.  Longer term, this project will expand to both incorporate public-key encryption protocols and software defined networking (SDN) capabilities to provide physical control of information routing.  We intend to provide public-key encryption capabilities via an integrated public key infrastructure providing additional privacy and non-repudiation abilities for the network and SDN capabilities via integration with OpenFlow.  Shorter term goals include inclusion of different modes of operation, so that the network can support both request/response and publish/subscribe modes of operation, and more robust development so the system can run as a commercial grade security-on-demand service.
