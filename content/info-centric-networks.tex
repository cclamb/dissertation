\section{Information-centric Networking}
\label{section:information-centric-networking}

%who what why how % when where
%
%I. What is it
%	Define it	
%	who they are
%	why are they important
%	what problem are they solving
%	introduce various different approaches
%		differentiate them

Information-centric networking is a new approach to internet-scale networks.  In general, it takes extensive advantage of data locality, cache data aggressively, decouple information providers from consumers, and use a content-centric perspective in network design. The overriding goals of this approach include providing higher information availability through better network resilience and implementing systems that more closely reflect today's use, focusing on heterogeneous systems from mobile to static access \cite{6231276}.  Four of the leading projects implementing these ideas are Data-Oriented Network Architecture (DONA) ~\cite{Koponen:2007:DNA:1282427.1282402}, Content-centric Networking (CCN) ~\cite{NDN,CCNx}, the Publish-Subscribe Internet Routing Project (PSIRP) ~\cite{PSIRP}, and the Network for Information (NetInf) ~\cite{NetInf}.

%		
%II. Common Features
%	why they have common features
%	Primitives and design approaches
%		contrast with past internet paradigms
%		current API structural commonalities (use IDL!)
%		how they would work
%		
%III. Differentiating Features
%	what are they
%	why are the different
%	what capabilities to the differences give us
%
%IV. Current Projects (just text, paragraph per, ref projects and survey paper)
%
%V. Conclude
%	impact if successful
%	what else? motivate importance of this work!
	