\section{Information-centric Networking}
\label{section:information-centric-networking}
Information-centric networking (ICN) is a new approach to internet-scale networks that shows promise with respect to decentralized, content-centric usage management, addressing scale and availability issues with current systems.  In general, it takes extensive advantage of data locality, caches data aggressively, decouples information providers from consumers, and uses a content-centric perspective in network design. The overriding goals of this approach include providing higher information availability through better network resilience and implementing systems that more closely reflect today's use, focusing on heterogeneous systems with requirements ranging from mobile to static access \cite{6231276}.  Four of the leading projects implementing these ideas are Data-Oriented Network Architecture (DONA) ~\cite{Koponen:2007:DNA:1282427.1282402}, Content-centric Networking (CCN) ~\cite{NDN,CCNx}, the Publish-Subscribe Internet Routing Project (PSIRP) ~\cite{PSIRP}, and the Network of Information (NetInf) ~\cite{NetInf}.  In general these projects and the thinking behind them is motivated by the belief that the current internet is not well suited to the way it is used today and that in order to efficiently support future use the internet needs to be fundamentally re-examined and perhaps, in some ways, re-implemented.

As expected, given that they are trying to solve similar problems and have taken similar conceptual approaches, the capabilities and features of these projects are similar as well.  To begin with, they all use named data objects as a central abstraction.  In this paradigm, information elements like videos, web pages, databases, and the like are represented by unique names rather than locations.  Current internet systems blur this distinction.  Uniform Resource Locators (URLs) were never designed to be used as content names, for example.  They are content addresses, describing the server and port on which content resides, the protocol to use to retrieve the content, and the specific location of that content on the identified server ~\cite{rfc3986}.  That said, they are still commonly used to name content, particularly in caching systems ~\cite{rfc2616} and content distribution networks ~\cite{Nygren:2010:ANP:1842733.1842736}.  The names of these data objects, since they have unique relationships to content, are tightly bound to the content they represent.  These names need to exhibit strong name-data integrity, so that the name can be trusted to refer to specific content, and the object retrieved must be verifiably authentic.  They have very similar programming interfaces.  These interfaces are built around acquiring and routing specific data objects from providers to consumers rather than forwarding bits from one system to another ~\cite{rfc791,rfc793}.  As a result, operations are oriented more toward registering interest in a named data object in some way, either through a specific object request or subscription, and the resulting delivery of that object ~\cite{6231276}.  ICN systems route information in similar ways as well, depending on the specific naming topology used as well.  Some ICN systems use name resolution services to bind specific objects to names ~\cite{4698847} while others use direct routing schemes to multiple hosts ~\cite{Ghodsi:2011:NCA:2018584.2018586}.  Finally, data objects are frequently cached on devices, both on edge devices and in-network.  These caches are generic and usable by any other services distributed throughout the network ~\cite{6231276}.

Contrast these design principles with those used to build the current internet.  Where internet-scale networks were originally designed in accordance with end-to-end principles and packetization of information concepts ~\cite{Cl:88,SaReCl:84}, ICNs are build with a much more data-centric perspective, focusing on routing information rather than just ones and zeros.  This is not a refutation of end-to-end arguments however.  Rather, it is an affirmation that some services must be incorporated into the fabrics of these kinds of systems reflecting the ubiquitous need for those functions ~\cite{BlCl:01}.

Though the various different types of ICNs certainly have similarities, they have significant differences too.  To begin with, they are not all synchronous.  PSIRP, for example, uses asynchronous publish/subscribe style communication between nodes, where requests and responses can be routed differently.  This is very useful for longer-lived sessions on mobile networks where the content is available at some point in the future, for example.  Furthermore, while most of these systems use the data object as the primary networking primitive, CCNs do not.  Rather, they still use packets.  They all use IP as the basic transport layer technology, but they can use other protocols as well.  Names may or may not be human-readable, and some systems use public-key infrastructures to enforce name-data integrity ~\cite{6231276}.

These differences can provide significant advantages.  Avoiding public-key cryptography in data object names allows for human readable names, for example ~\cite{6231276}.  Allowing network clients to subscribe to content allows dynamic content to be generated and then routed to a client at some point in the future.  This is still an active area of research where the most appropriate solutions have yet to be established.

The fact that these systems have independently concluded that future macro-networks should use a data object as the primary abstraction is significant in that it demonstrates a widespread belief in the approach.  Furthermore, application-level overlay networks have used this essential approach to content management as well, though not as pervasively as ICNs ~\cite{Tarkoma:2010:ONT:1805887}.  These types of solutions do offer significant advantages and solve many of the problems currently facing large-scale networks ~\cite{6231276}.  Timelines for adoption, feasibility and form of migration, and other important issues have yet to be established at this point.  With that in mind, commercial organizations have begun to invest research and development budgets in this kind of work ~\cite{xerox,parc}.  

	