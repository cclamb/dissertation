\section{Characteristics}
\label{section:characteristics}
Examining content at each network node or router point can certainly impact performance and by extension availability.  It is also important to establish that this kind of dynamic dispatch will actually guarantee delivery along the most secure path needed.  With respect to delivery, it need to be shown that by selecting optimum paths at given network points the overall selected path will have the appropriate security characteristics outlined by any policy associated with delivered content.

To begin with, imagine in a given aggregate path between two points if local decisions are made with respect to routing based on specific security criteria at interleaving points the path as a whole will adhere to those security criteria.  Essentially, this implies that it is possible to use a greedy algorithm with respect to security and routing and that the algorithm will yield an optimal security path.  It is important to recognize that this is key to establishing a secure route between two specific points.  Furthermore, in a given route, that route must be viewed temporally as well, in that each link may not be optimal when the delivered data element reaches a destination, but each link was optimal at the time it was selected, and by extension, when the aggregate path is reviewed, it would likewise be optimal with respect to time of traversal.  Finally, local nodes may very well have knowledge about the local environment that cannot be known by a centralized routing authority.  Allowing local routing decisions with respect to security can help take advantage of this locality.

The idea behind the proof is if a path $P$ exists consisting of nodes and edges $\lbrace V, E \rbrace$, that path was then assembled by choosing the most secure edge $e \in E$ from a corresponding node $v \in V$ at some time $t$.  The path $P$ consisting of these edges $e$ is then the most secure path that can be chosen.  A proof by contradiction establishes the feasibility of this approach:

\begin{enumerate}
\item Assume a path $P = \lbrace V, E \rbrace$.
\item Assume that $P$ is not the most secure, and that a more distinctly secure path $P' = \lbrace V, E' \rbrace$ exists.
\item If $P'$ exists, then at some $v \in V$ $\exists$ $e' \in E$ such that $e'$ is more secure than the corresponding edge $e \in E$.
\item If so, then at all$v \in V$, $e' = e$ leading to $E' = E$ and $P' = P$, so $P'$ is not distinct from $P$. 
\end{enumerate}

This assumes that a path of some kind does exist.  If so, and if the most secure edge is chosen at each node, the resulting path will in fact be appropriately secure and policy compliant.