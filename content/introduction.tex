Current enterprise computing systems are facing a troubling future.  As things stand today, they are too expensive, unreliable, and information dissemination procedures are just too slow.  Current approaches to partitioning information in cross-domain scenarios are simply unable to migrate to cloud environments because of reliance on control of physical hardware to enforce information separation.  The current approach of controlling information by controlling the underlying physical network is too cost ineffective to continue, and does not scale into shared datacenters ~\cite{RedBook}.  This leaves large government and commercial organizations concerned with avoiding the exposure of sensitive data in a very uncomfortable position, where they cannot continue doing what they have done, and cannot migrate to what everyone else is doing.

Generally, systems handling sensitive information still do not use current commercial resources as well as they could and use costly data partitioning schemes.  Most of these kinds of systems are managed in house by the enterprise itself rather than exploiting lower cost cloud-enabled services.  Furthermore, many of these systems have large maintenance loads imposed on them as a result of internal infrastructural requirements like data and database management or systems administration.  In many cases networks containing sensitive data are separated from other internal networks to enhance data security at the expense of productivity, leading to decreased working efficiencies and increased costs.

These kinds of large distributed systems suffer from a lack of stability and reliability as a direct result of their inflated provisioning and support costs.  Simply put, the large cost and effort burden of these systems precludes the ability to implement the appropriate redundancy and fault tolerance in any but the absolutely most critical systems ~\cite{Tallon:2010:UDI:1735223.1735253}.  Justifying the costs associated with standard reliability practices like diverse entry or geographically separated hot spares is more and more difficult to do unless forced by draconian legal policy or similarly dire business conditions.

Finally, the length of time between when a sensitive document or other type of data artifact is requested and when it can be delivered to a requester with acceptable need to view that artifact is prohibitively long.  These kinds of sensitive artifacts, usually maintained on partitioned networks or systems, require large amounts of review by specially trained reviewers prior to release to data requesters.  In cases where acquisition of this data is under hard time constraints like sudden market shifts or other unexpected conditional changes this long review time can result in consequences ranging from financial losses to loss of life.

Federal, military, and healthcare computer systems are prime examples of these kinds of problematic distributed systems, and demonstrate the difficulty inherent in implementing new technical solutions.  They, like other similar systems, need to be re-imagined to take advantage of radical market shifts in computational provisioning.  New approaches to networking and information management present possible solutions to these kinds of problems by providing distributed information-centric approaches to data management and transfer ~\cite{proposal:info-sharing-strategy}.

Current policy-centric systems are being forced to move to cloud environments and incorporate much more open systems.  Some of these environments will be private or hybrid cloud systems, where private clouds are infrastructure that is completely run and operated by a single organization for use and provisioning and hybrid clouds are combinations of private and public cloud systems.  Driven by both cost savings and efficiency requirements, this migration will result in a loss of direct control of computing resources by involved organizations as they attempt to exploit economies of scale and utility computing.

Robust usage management will become an even more important issue in these environments.  Federal organizations poised to benefit from this migration include agencies like the United States National Security Agency (NSA) and the United States Department of Defense (DoD), both of whom have large installed bases of compartmentalized and classified data.  The DoD realizes the scope of this effort, understanding that such technical change must incorporate effectively sharing needed data with other federal agencies, foreign governments, and international organizations ~\cite{proposal:info-sharing-strategy}.  Likewise, the NSA is focused on using cloud-centric systems to facilitate information dissemination and sharing ~\cite{proposal:nsa-cloud}.

Cloud systems certainly exhibit economic incentives for use, providing cost savings and flexibility, but they also have distinct disadvantages as well ~\cite{proposal:privacy-security-trust-cloud}.  How to address these issues is an open research question.  Organizations ranging from cloud service providers to the military are exploring how to engineer solutions to these problems, and to more clearly understand the trade-offs required between selected system architectures ~\cite{proposal:assured-info-sharing}.  The problems themselves are wide ranging, appearing in a variety of different systems.  Military and other government systems are clearly impacted by these kinds of trust and security issues, and they also have clear information sensitivity problems.  This, coupled with the fact that these organizations have been dealing with these issues in one form or another for decades make them very well suited for prototypical implementation and study.

This chapter will cover national and international standards in this area, current solutions in place to address some of these problems, the state of the art in information networks, and other related work.  Organizations have been trying to standardize security approaches since the 1980's, starting with the notorious Orange Book ~\cite{OrangeBook}, and ending with today's NIST cloud standards ~\cite{NIST800:144}.  Information-centric networks are a new approach to information management that promises more efficient content  management and supplies new capabilities for information security. The DoD has been key in some of these efforts, and continues to play such a role today, demonstrated in the state of the art in today's cross-platform solutions.  The second chapter introduces and analyzes a proposed architectural taxonomy to address the information sharing goals held by the DoD through the Unified Cross Domain Management Office (UCDMO), and the third describes in detail the development of this prototypical system, starting with a single-system proof-of-concept and working through the current nation-spanning cloud network.  The final chapter covers specific experimental results and analysis of these techniques from a confidentiality, integrity, and availability perspective.
