\section{Confidentiality and Integrity Models}
Current models currently in active use to support information confidentiality and integrity include the Bell-LaPadula model, the Biba model,  the Clark-Wilson model, and the Brewer and Nash model.  Of these, Bell-LaPaudula and Brewer and Nash address information confidentiality, while Biba and Clark-Wilson address information integrity.

Bell-LaPadula was developed in the 1970's to address information confidentiality in the centralized mainframe and minicomputer environments  common in military installations of the day.  It is essentially a mathematical state machine model that establishes rules with respect to how information can flow in stratified environments.  It is established around the {\it Basic Security Theorem}, which essentially states that a system with a secure initial state and only secure transitions is guaranteed to terminate in a secure final state.  It extends this theorem to establish four rules.  The first is the {\it simple security rule}.  The simple security property prohibits reading information from security levels higher than that occupied by the reader.  The next two are the {\it *-property rule} and the {\it strong *-property rule}, which prohibit writing data to any security level less than that of the writer.  The final property is the {\it ds-property}, or {\it Discretionary Security Property}.  The ds-property allows a subject to pass permissions on to other subject's at the initial subject's discretion.  This model also introduces the concepts of {\it dominance relations} and {\it tranquility}.  Dominance relations exist within partially ordered sets of classification levels, where higher classification levels are said to dominate lower levels.  This means that if a subject is cleared to access top secret material, that subject can also access secret and confidential material, as both secret and confidential material is considered to be less sensitive that top secret material.  Tranquility limits object state changes after creation.  Essentially, any object is created at a specific sensitivity level and that sensitivity level does not change with time ~\cite{Bell1973}.

The Brewer and Nash model is currently widely used in the financial services industry.  Also referred to as the Chinese Wall Model, this model changes what subjects can access based on a subject's context.  In this model, information is continually monitored so that no subject is allowed access to an object that may create a conflict of interest or other moral hazard.  More generally, access to objects is dynamically evaluated based on the context of a given subject's previous object accesses. This enforces information confidentiality based on the context of information access ~\cite{Brewer89}.

The Biba model, developed in the late 1970's, addresses information integrity using a layered approach similar to that used by Bell-LaPadula, and was in fact developed to complement Bell-Lapadula's focus on confidentiality.  Bell-LaPadula features a hierarchy of information sensitivity that guides access decisions.  Biba, on the other hand, is based on a hierarchy of information integrity rather than confidentiality.  Biba also has a collection of specific properties that must be adhered to.  The first, the {\it simple integrity axiom}, holds that a subject cannot access an object of a lower integrity level.  The second, the {\it *-integrity axiom}, likewise states that a subject cannot write information to higher integrity level.  The third property, the {\it invocation property}, states that subjects cannot invoke objects at higher integrity levels either ~\cite{Biba1977}.

David Clark and David Wilson presented their integrity model in 1987 to address business information integrity as Biba was thought to be more suitable for military use.  The Clark-Wilson model uses five essential components.  The first, {\it users}, are active subjects requesting access to objects.  The second, {\it transformation procedures}, are operations allowed over objects.  These general operations are any that may change the state of an object, like create, update, and delete operations, as well as simple read access.  Systems are also required to have collections of {\it integrity verification procedures} which validate the state of managed objects.  The object themselves have two distinct flavors, {\it constrained data items} and {\it unconstrained data items}.  Constrained data items are those managed via trusted transformation procedures and monitored via integrity verification procedures.  Unconstrained items can be directly accessed.  Access and management of data items are constrained by a set of nine rules, divided into sets of certification and integrity rules.  In essence, these rules ensure that all constrained data item access is through trusted transformation procedures, and that the trusted procedures ensure that information is maintained in a valid state.  Subjects are limited to sets of trusted transformation procedures expressed by $(subject, procedure, object)$ triples, while unconstrained data items can be accessed without limitation ~\cite{ClaWil87}.





